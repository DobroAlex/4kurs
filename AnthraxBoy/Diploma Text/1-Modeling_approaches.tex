\newpage
\parindent=1cm %красная строка
\begin{center}
		
		\section{Подходы к моделированию процесса распространения инфекционных заболеваний}
		
\end{center}

Эпидемии всегда являлись одной из наиболее острых проблем медицины. Внезапно возникающие эпидемии формируют сложную ситуацию, динамически изменяющуюся во времени. В таких условиях медики и связанные чрезвычайные службы зачастую не могут принять адекватного решения и только порождают дополнительный хаос, неэффективно расходуют ресурсы, не способны в целом положительно повлиять на ход эпидемии или реализовать меры по борьбе с заболеванием. 


Логично предположить, что одним из способов снижения угрозы для населения и уменьшения хаоса среди медиков и прочих ответственных служб являются меры предупредительного характера. Предупредительные меры могут иметь двойственный характер: 
\begin{enumerate}
	\item Во-первых, на основании регулярно обновляемых статистических данных, описывающих эпидемиологическую обстановку, возможно создавать кратковременную прогнозирующую модель, что позволит обнаружить эпидемию ещё до её начала;
	\item Во-вторых, учет данных о прошлых подобных вспышках заболеваемости позволит составить прогноз для ответственных служб, описывающий характер необходимых ресурсов, их количество и область применения, что в свою очередь позволит уменьшить порождаемый вспышкой хаос.
\end{enumerate}

Подобная идея не является революционно новой и  к моделированию распространения заболеваний существует  достаточно много подходов, основанных на разных принципах: цепи Маркова, агентный подход, стохастические эксперименты, дифференциальные уравнения на графах, различные алгоритмы на графах. Далее будут подробно рассмотрены некоторые из этих подходов.

