\newpage
\parindent=1cm %красная строка
\begin{center}
		
		\section{Подходы к моделированию процесса распространения инфекционных заболеваний}
		
\end{center}

Эпидемии всегда являлись одной из наиболее острых проблем медицины. Внезапно возникающие эпидемии формируют сложную ситуацию, динамически изменяющуюся во времени. В таких условиях медики и связанные чрезвычайные службы зачастую не могут принять адекватного решения и только порождают дополнительный хаос, неэффективно расходуют ресурсы, не способны в целом положительно повлиять на ход эпидемии или реализовать меры по борьбе с заболеванием. 


Логично предположить, что одним из способов снижения угрозы для населения и уменьшения хаоса среди медиков и прочих ответственных служб являются меры предупредительного характера. Предупредительные меры могут иметь двойственный характер: 
\begin{enumerate}
	\item На основании регулярно обновляемых статистических данных, описывающих эпидемиологическую обстановку, возможно создавать кратковременную прогнозирующую модель, что позволит обнаружить эпидемию ещё до её начала;
	\item Учет данных о прошлых подобных вспышках заболеваемости позволит составить прогноз для ответственных служб, описывающий характер необходимых ресурсов, их количество и область применения, что в свою очередь позволит уменьшить порождаемый вспышкой хаос.
\end{enumerate}

Подобная идея не является революционно новой и  к моделированию распространения заболеваний уже  существует  достаточно много подходов, основанных на разных принципах: цепи Маркова, агентный подход, стохастические эксперименты, дифференциальные уравнения на графах и другие  различные алгоритмы на графах, полигамные модели. Далее будут подробно рассмотрены некоторые из этих подходов.

Некоторые общие допущения для всех моделей:
\begin{itemize}
	\item Равномерность и стационарность в смысле неизменяемости во времени  распределения возрастов, то есть все в популяции живут до некоторого возраста $\mathit{L}$ и затем умирают и для каждого возраста (включительно до $\mathit{L}$) количество людей в этой возрастной группе примерно или строго равно. Этот подход хорошо подходит для экономически развитых стран и стран, находящихся на постиндустриальном этапе экономического развития, где детская смертность мала и большинство людей доживают до ожидаемого возраста. Данное предположение может быть отвергнуто для стран, не подходящих под условия выше, например, при моделировании распространения лихорадки Эбола в странах Центральной и Южной Африки;
	\item Гомогенность или однородность перемещений внутри популяции, то есть индивидуумы  в популяции контактируют случайно, не замыкайся в более мелких подгруппах. Данное допущение является спорным, т.к социальная структура широкомасштабная и сложная, то есть индивидуумы внутри одной большой группы могут находиться в таких социальных отношениях, что большинство контактов будет приходиться на их собственную подгруппу, в то время как количество контактов вне группы будет предельно малым. Однако, такое допущение имеет место быть для упрощения построения модели   и понимания результатов моделирования.
\end{itemize}


Также опишем один общий принцип разделения моделей по признаку случайности. По такому признаку  модели можно разделить на стохастические и детерминистические. Стохастические модели предполагают наличие случайных величин  и являются инструментом для оценки распределения вероятности потенциальных исходов  посредством допущения случайных изменений в одной или нескольких переменных с ходом времени. Такие модели зависят от случайных изменений рисков, связанных с продолжительностью экспозиции, вероятности заболевания и прочих динамических изменяющихся параметров заболевания. Для описания различных этапов заболевания часто используют буквы M(maternally-derived immunity), S(suspicious / suspected), E(exposed), I(infected), R(recovered).

\subsection{Агентный подход к имитационному моделированию процесса распространения заболевания}

Существует несколько понятий определения "агент" и "агентный подход" в целом. Обобщая их, можно выделить, что  \textit{агент}  -- это некоторая сущность, имеющая активность, автономное поведение, способная самостоятельно принимать решения в соответствии с некоторой совокупностью правил, взаимодействовать с окружающей средой, если такая предусмотрена, и другими агентами, если такие существуют. Основное поле применения таких моделей -- децентрализованные систем, динамика функционирования которых не является следствием некоторых, зачастую внешних, правил и законов, но наоборот, такие правила и законы являются внутренним неотъемлемым результатом работы множества агентов. Такие модели обычно являются  дискретно-событийными или дискретными с непрерывными элементами, то есть гибридными. 

Данный подход удобно иллюстрировать на примере моделирования распространения гриппа (обычно, гриппа А), т.к грипп вносит весомый вклад в смертность при инфекционных заболеваниях и при этом хорошо известны его способы передачи и формы течения болезни. Также по данному заболеванию собрана достаточно обширная статистика, который можно использовать для проверки точности модели и последующей калибровки.  Такая модель достаточно подробно рассмотрена в  %REF: М.А. Кондратьева, Р.И. Ивановского, Л.М. Цымбалова <<Применение агентного подхода к имитационному моделированию процесса распространения заболевания>> 
\cite{Kondratiev:2010}
и описывает модель, способную определять число больных гриппом в каждый день определенного краткосрочного периода (2-3 недели) во время сезонных вспышек, то есть построение краткосрочного прогноза. Для этого популяция  разбивается на возрастные группы, позволяющие характеризовать контактов с другими людьми в сутки и, что важно, потенциально посещаемые места. В работе вводятся два основных типа объектов: локации и агенты и связанные с ними события перемещения агентов между локациями, протекания заболевания у агентов и событие <<контакта>>. Также в данной работе вводятся правила, описывающие поведение агентов с помощью UML Statecharts (далее <<стейтчарты>>). С помощью таких стейтчартов описывается переход между различными состояниями индивидуума как вне течения болезни, так и после, и может быть обозначен с помощью графической нотации как: 

\begin{figure}
	\centering{\includegraphics[scale=0.5]{Images/DiplomaStates.png}}\label{UML_Diag_1}
	\caption{Пример записи возможных состояний агента с помощью UML Statecharts}
\end{figure} 


	



Данная модель обладает рядом преимуществ: высокая скорость разработки в силу выбора готового имитационного ПО (специфично для данной рассматриваемой работы, хотя в целом верно для разработки любой агентной модели в силу простоты инкапсуляции агентов и построения связей между объектами); высокая скорость работы, так как выбранная модель допускает высокий параллелизм в рамках одной итерации, когда вычисления для каждого объекта-места допускают распараллеливание; свобода от дифференциальных уравнений значительно упрощает расчеты, в том числе исчезает необходимость использовать численные методы и связанные с ними понятия <<сходимости>> и <<устойчивости>>, при этом последнее особенно важно, так как входные статистические данные зачастую имеют шумы;  сам агентный подход допускает быструю модификацию модели и возможность учета применяемых административных мер  и их влияния на ход распространения инфекции.

Недостатки: в силу собственной стохастичности, модель требует многократного запуска  и оценкой экспертом полученных результатов для получения некоторого <<среднего результата>>; достаточно сложная модель может потребовать много времени на разработку даже при использовании готового имитационного ПО, при этом такая модель может оказаться ресурсозатратной в терминах машинных мощностей (время, память); невозможно точно определить, насколько точна данная модель по отношению к реальным, кроме как посредством многократного сравнения результатов работы модели и собранных существующих данных, при этом невозможно убедиться, что данная модель и текущие параметры могут быть применимы для прогнозирования, а не только откалиброваны для соответствия предыдущим результатам; стохастическая природа модели при компьютерной реализации опирается на генерацию случайных чисел, что требует использования мощных и проверенных генераторов. 

\subsection{Применение случайных графов для распознавания и анализа инфекционных заболеваний}

	В качестве другого подхода к моделированию распространения инфекционных заболеваний можно рассмотреть переносчиков заболеваний как  саморазмножающиеся сущности, т.е такие сущности, которые способны самореплицироваться, при этом свойство  саморепликации передается не только между <<родителями>> и <<потомками>> (вертикальный перенос), но возможен и горизонтальный перенос, т.е передача свойства саморепликации и, возможно, но не обязательно, других свойств, объектам, которые так или иначе соседние или контактируют с данным.  Примерами являются размножение компьютерных вирусов и сетевых червей  %REF: [1] с https://cyberleninka.ru/article/v/sluchaynye-grafy-kak-model-sredy-rasprostraneniya-i-vzaimodeystviya-samorazmnozhayuschihsya-obektov
	\cite{Bratus:2010}
 или инфицирование организмов вирусами и инфекциями %REF: [3;5] оттуда же 
 \cite{Klimentiev:2013}, \cite[с. 133]{Smith:Zombies}
. 
В данных моделях обычно применяются графы, используемые для моделирования эпидемий <<мобильных червей>> (т.е вредоносных программ, распространяющихся между устройствами при помощи беспроводных протоколов типа Bluetooth или Wi-Fi, широко применяемых в мобильных устройствах, отсюда и происходит название) %REF: http://www.ssc.smr.ru/media/journals/izvestia/2016/2016_4_744_748.pdf стр 744
%REF:  КЛИМЕНТЬЕВ К.Е. Компьютерные вирусы и антивирусы: взгляд программиста Москва: ДМК-Пресс, 2013. 656с. стр 27
\cite{Klimentiev:2016}, \cite[с. 27]{Klimentiev:2013}
или воздушно-капельных инфекций среди высших  животных.   %REF: [1] https://cyberleninka.ru/article/v/sluchaynye-grafy-kak-model-sredy-rasprostraneniya-i-vzaimodeystviya-samorazmnozhayuschihsya-obektov
Согласно К.Е. Климентьеву %REF: https://cyberleninka.ru/article/v/sluchaynye-grafy-kak-model-sredy-rasprostraneniya-i-vzaimodeystviya-samorazmnozhayuschihsya-obektov
\cite{Klimentiev:2015}, для подобных эпидемий характерны следующие черты:
\begin{itemize}
	\item Постоянные изменения топологии среды моделирования в связи с высокой мобильностью  объектов;
	\item ограниченный радиус <<заражения>>, как для биологических агентов, так и для компьютерных вирусов, обусловленный физическими свойствами оных.
\end{itemize}


Для моделирования среды существования таких объектов используется класс сетей, называемый <<специальным>> (<<Ad hoc>>), представляющий собой множество случайных графов с разнообразной топологией. При этом нетрудно выделить общие для таких графов характеристики: 
\begin{itemize}
	\item Вероятностное распределение степеней вершин, обозначаемое $k_i$;
	\item также вероятностное распределение для локальных коэффициентов кластеризации вершин, обозначаемое $c_i$.
\end{itemize}
При этом под <<степенью вершины>> 	$x$ подразумевается количество вершин графа $G$, инцидентных вершине $x$, т.е степень вершины указывает количество вершин, непосредственно соединенных только один ребром с данным %REF: Дистель, Рейнхард (2005), Graph Theory (3rd ed.), Berlin, New York: Springer-Verlag, ISBN 978-3-540-26183-4. , стр 5
\cite[с. 5]{DiestelR:2005} .
<<Локальный 	коэффициент кластеризации вершины>> (ЛККВ) понимается в том же смысле, что и у К.Е. Климентьева и Б. Хогана %REF : https://cyberleninka.ru/article/v/sluchaynye-grafy-kak-model-sredy-rasprostraneniya-i-vzaimodeystviya-samorazmnozhayuschihsya-obektov
%REF: Hogan, B., Carrasco, J., & Wellman, B. (2007). Visualizing personal networks: Working with participant aided sociograms. Field Methods, 19(2), 116-144.
\cite{Klimentiev:2015}, \cite{Hogan:2017}, т.е <<коэффициент кластеризации>> рассматривается как вероятность того, что два ближайших соседа этого узла сами являются ближайшими соседям, а ЛККВ является мерой того, насколько хорошо связанны связаны между собой соседи данного узла и рассчитывается как отношение
числа связей межу соседями данного узла к возможному числу связей между соседями  (по Хогану) или, менее формально, по К.Е Климентьеву: <<доля <<треугольников>> данной вершины, образованных из <<соседей>> данной вершины и являющихся <<соседями>> друг для друга в общем количестве потенциально возможных треугольников, где $c_{max}=k_i\dfrac{k_i-1}{2}$ >> \cite{Klimentiev:2015}.


В качестве модели пространства, используемых для имитации развития и распространения эпидемии, применяются неориентированные, маркированные графы (в смысле частного случая сети Петри, в которой каждая позиция входом и выходом точно для одного перехода %REF: http://publ.lib.ru/ARCHIVES/P/PITERSON_Djeyms/_Piterson_Dj..html, стр 200) 
\cite[с. 200]{Piterson:1981} ), являющиеся частным случаем дистанционных графов, при этом дистанционный граф понимается как $G = (V,E)$ -- n-мерный дистанционный граф  (граф расстояний), если $V \subseteq \mathbb{R}^n, E \subseteq \{\{\overline{x}, \overline{y} \}: \overline{x}, \overline{y} \in V, |\overline{x} - \overline{y}| = a, a \in \mathbb{R} > 0 \}$, то есть множество вершин $M$ является подмножеством или совпадает с  $n$-мерным пространством,а множество ребер $E$ является подмножеством или совпадает с множеством всевозможных пар вершин $\overline{x}, \overline{y}$, таких, что  евклидово расстояние между $\overline{x}, \overline{y}$ равно некоторому фиксированному вещественному положительному заранее заданному $a$. 

Топология таких графов связана с принципом их построения, т.е каждая вершина графа имеет некоторые пространственные координаты (обычно в $\mathbb{R}^2$ или $\mathbb{R}^3$) и <<соседними вершинами>> считаются только те, расстояние до которых  меньше заданного или равно заданного $r_0$ (расстояние передачи вирусного воздействия). 

Допущения:
\begin{itemize}
	\item Инфекция передается только между инцидентными вершинами за некоторое время;
	\item вершины графа активно перемещаются, постоянно изменяя его конфигурацию.
\end{itemize}
Входные параметры модели:
\begin{itemize}
	\item Начальная конфигурация графа, т.е количество и координаты вершин вместе с их начальным состоянием, обычно в виде перечисления 	<<здоров>>, <<вакцинирован>>, <<болен>>, однако допускается и большее число состояний, подобное применяемому в агентных подходах;
	\item максимальное расстояние инфицирующего воздействия $r_o$;
	\item правила или порядок перемещения вершин в процессе моделирования.
\end{itemize}

Способы построения исходного графа, исходя из <<геометрических>>  и <<географических>> по К.Е Климентьеву соображений подробно рассмотрено в %REF: http://climentieff.ssau.ru/download/Climentieff_Ufa_2012.pdf 
 \cite{Klimentiev:2012} и %REF: https://cyberleninka.ru/article/v/sluchaynye-grafy-kak-model-sredy-rasprostraneniya-i-vzaimodeystviya-samorazmnozhayuschihsya-obektov
\cite{Klimentiev:2015}. При этом для построения геометрического графа по заранее заданным параметрам типа $k_i$ и $c_i$ используется случайный граф Радо (RRG) или случайный геометрический граф (RGG) и в этой же работе %REF:https://cyberleninka.ru/article/v/sluchaynye-grafy-kak-model-sredy-rasprostraneniya-i-vzaimodeystviya-samorazmnozhayuschihsya-obektov
\cite{Klimentiev:2015} выводится функция распределения  между двумя случайными точками в единичном квадрате и математическое ожидание расстояния.

Намного более интересным свойством данных моделей являются правила перемещения вершин графа: если в описанных ранее подходах перемещения вершин присутствуют скорее в силу необходимости для динамичности во времени и модель учитывает только их положение по отношению к другим вершинам и, иногда, скорость движения этих, а топология графа не является предметом исследования, то в данном подходе топологию графа и правила перемещения являются ключевым  элементом  при построении модели. В результата построения множества моделей, использующих данный подход, можно сделать следующие выводы, описывающие правила перемещения вершин в графе:
\begin{itemize}
	\item Классические модели блуждания, описывающие, например,  броуновское движение, мало пригодны для описания передвижения людей, особенно движущихся вне рутинного маршрута, а высокоподвижных агентов, в силу своей слабой стационарности. Следовательно, при анализе  движений по плоскости, направление движения распределено равномерно на интервале $[0; 2\pi]$, что показано в \cite{Rhee:2007} %REF: Rhee I. et al. On the Levi Walk Nature of Human Mobility: Do Humans Walk Like Monkeys // IEEE/ACM Transaction on Networking, Vol. 20.	- pp. 630-643.
	;
	\item скорость и продолжительность движения распределены по Леви. При этом <<перемещение>> или <<движение>> понимается как  самый длинный прямолинейный  переход объекта с одного места в другое без паузы или изменения направления %REF: http://repo.ssau.ru/bitstream/Informacionnye-tehnologii-i-nanotehnologii/Modelirovanie-peredvizhenii-uzlov-DTN-seti-s-ispolzovaniem-principa-naimenshego-deistviya-pri-vybore-lokacii-posesheniya-62586/1/itnt_2015_61.pdf
	\cite{Privalov:2015};	
	\item тем не менее, перечисленные правила перемещения актуальны только для перемещениям на ограниченном участке плоской(или приближенной к плоской)  поверхности без учета препятствий, конфигураций помещения, уставившихся маршрутов, столкновений с другими телами и необходимости их избегать. Для внесения корректировок, учитывающих такие особенности, необходим сбор статистики на множестве реальных наблюдений с последующим внесением в некоторую  ГИС. 
\end{itemize} 

Базируясь на работах %REF: Rhee I. et al. On the Levi Walk Nature of Human Mobility: Do Humans Walk Like Monkeys // IEEE/ACM Transaction on Networking, Vol. 20.	- pp. 630-643.
\cite{Rhee:2007},
%REF: http://repo.ssau.ru/bitstream/Informacionnye-tehnologii-i-nanotehnologii/Modelirovanie-peredvizhenii-uzlov-DTN-seti-s-ispolzovaniem-principa-naimenshego-deistviya-pri-vybore-lokacii-posesheniya-62586/1/itnt_2015_61.pdf
\cite{Privalov:2015} ,
%REF: http://climentieff.ssau.ru/download/Climentieff_Ufa_2012.pdf
\cite{Klimentiev:2012},
%REF: http://www.ssc.smr.ru/media/journals/izvestia/2016/2016_4_744_748.pdf стр 744
\cite{Klimentiev:2016} можно сделать следующие выводы, касающиеся основных свойств данного подхода к моделированию: 
\begin{enumerate}
	\item Данный подход хорошо подходит для моделирования распространения инфекционных заболеваний среди групп людей и животных, особенно блуждающих, в т.ч. по Леви, в некотором постоянном данном ареале;
	\item использование данных моделей улучшает понимание того, какие графы лучше описывают топологию быстро движущихся источников инфекции. Так, в работах %REF: http://climentieff.ssau.ru/download/Climentieff_Ufa_2012.pdf 
	\cite{Klimentiev:2012} и % REF: https://cyberleninka.ru/article/v/sluchaynye-grafy-kak-model-sredy-rasprostraneniya-i-vzaimodeystviya-samorazmnozhayuschihsya-obektov 
	\cite{Klimentiev:2015} четко прослеживается зависимость между <<Ad hoc>> графами и графами Радо и той картиной, которую дают реальные перемещения мобильных устройство при вспышке компьютерных инфекций.
\end{enumerate}


Достоинства данного подхода: высокая скорость работы моделей при правильном выборе инструментария(готовые фреймворки для работы с графами); высокая точность предсказания протекания вспышки инфекции на начальном этап этой вспышки; наличие уже разработанных алгоритмов, приборов и готовых собранных данных для занесения в многослойные ГИС; возможность модификации алгоритмов для учета рельефа местности и возможных столкновений инфицируемых  сущностный; математический аппарат блужданий Леви допускает модификации, позволяющие легко настраивать и изменять принципы перемещения сущностней; сопоставление результатов моделирования с реальными процессами позволяет изучать зависимости между некоторыми специальными графами, описанными выше, и путями распространения инфекций. 


Недостатки: в силу специфики требований, предъявляемых к моделям данного типа, точное моделирование возможно только в течении небольшого количества итераций модели, т.к предполагается характер <<вспышки>>, то есть быстрого бурного увеличения числа зараженных индивидуумов с достаточно быстрым падением числа больных при изобретении лекарства или разработке антивируса; для построения точных моделей требуется не только точно знать координаты физического пространства, на котором будет происходить распространения заболевания, но и иметь детальные модели самих поверхностей или рельефов, что необходимо для точного построения блужданий Леви; используемые графы высокодинамичные и могут требовать значительных затрат как по памяти, так и по процессорному времени.

\subsection{Модели инфекционной динамики на основе предфрактальных графов}

Для моделирования распространения инфекционных заболеваний можно также использовать предфрактальные графы.В основу данного подхода к моделированию положена идея о том, что графы, использующие операцию "замены вершины затравкой" (ЗВЗ), которая будет рассмотрена далее, <<прирастают>> не отдельными вершинами, а самоподобным или приближенно самоподобным  графом, т.е фрактальным, хорошо описывают   системно <<прирастающие>> структуры: телекоммуникационные, инфраструктурные, социальные и технические сети, некоторые популяции, в которых существует разбиение на группы, в которых связей внутри группы больше, чем между отдельными группами. При этом показано,  что такие модели могут быть использованы для моделирования распространения инфекционных заболеваний, а именно, что процессы разрастания  графов релевантны динамике распространения инфекционных  заболеваний  и распад таких графов соответственно релевантен  спаду эпидемий %REF: https://new-disser.ru/_avtoreferats/01005400379.pdf 
\cite{Utakaeva_disser:2011}
Математико-алгоритмическая  идея построения фрактальных графов -- операция замены вершины затравкой (ЗВЗ). Для этого вводится несколько понятий, рассмотренных ниже.
По определению, динамический граф ${G_D}$ -- последовательность обычных графов $G_l$, не имеющих параллельных ребер и петель. Затравка -- какой-либо произвольный связный граф, т.е граф, в котором между любой парой вершин существует хотя бы один путь, сама затравка обозначается как  ${H=(W,Q)}$. Предфрактальный граф обозначается как $G_L=(V_L,E_L)$, определяется рекуррентно, заменяя каждый раз в построенном на предыдущем этапе $l = \overline{1, L-1}$ графе $G_l=(V_l,E_l)$ каждую вершину затравкой $H$. Вершины могут соединяться случайно или по заданному правилу или в некоторой заранее заданной последовательности. На этапе $l=1$ предфрактальному графу соответствует затравка $G_1=H$. Говорят, что предфрактальный граф $G_L$  \textit{порожден} затравкой $H$, при этом процесс построения предфрактального графа $G_L$  по сути является процессом построения последовательности других предфрактальных графов $\overline{G_1,G_L}$, называемый <<траекторией>>. 

Суть операции замены вершины затравкой (ЗВЗ): в графе $G=(V, E)$ из выбранной для замещения затравкой вершины $\tilde{v}  \in V $ выделяется множество $\tilde{V} = \{\tilde{v}_j\} \subseteq V$, где j = $\overline{1,|\tilde{V}|}$, смежных ей, вершине $\tilde{v}  \in V $, вершин; затем из графа $G$ удаляется вершина $\tilde{v}$ и все инцидентные ребра, т.е ребра, для которых эта вершина общая; после $\forall$ $\tilde{v}_j \in \tilde{V}, j=\overline{1,|\tilde{V}|}$ соединяется ребром с одной из вершин затравки $H$, при этом соединение может происходить случайно или по заранее заданному правилу или в заданной последовательности.

Под <<распознаванием предфрактального графа>> в таких подходах к моделированию  обычно понимается  определение траектории построения предфрактального графа при условии, что заданы виды и типы затравок. При этом задача распознавания различных типов предфрактальных графов хорошо изучена и отработана в практических алгоритмах, что дает широкую теоретико-практическую базу для решения данной задачи %REF: https://cyberleninka.ru/article/v/raspoznavanie-predfraktalnyh-grafov-s-zatravkoy-udovletvoryayuschey-usloviyu-ore
\cite{Reznikov:2010},
%REF: Найманова И.Х., Кочкаров А.М. Об одной задаче распознавания предфрактального графа // Вестник Самарского государственного технического университета. - 2007. - № 1 . - С. 194-196; 
\cite{Naimanova:2007},
%REF: Утакаева И.Х. Алгоритм распознавания предфрактального графа с затравкой регулярной степени // Обозрение прикладной и промышленной математики. - 2008. -Том 15.- Выпуск3. - С. 531-533;
\cite{Utukaeva:2008},
%REF: Утакаева И.Х., Кочкаров А.М. Моделирование процесса распространения эпидемии и нахождения возможных очагов заражения на предфрактальном графе // Сборник трудов 111-ей Всероссийской научно-практической конференции «Перспективные системы и задачи управления». - Таганрог: Издательство Таганрогского технологического института ЮФУ, 2011. - С.273-283 
\cite{Utukaeva:2011}
.

В предфрактальном графе $G$ рёбра, порожденные на этапе $l$, где $l \in {1,2,3, ..., L}$ называются ребрами ранга $l$, при этом <<новыми>> ребрами называют ребра ранга $L$ в предфрактальном графе $G_L$, все остальные называются <<старыми>>.  При этом, под рангом вершины графа понимается наименьший ранг $l$, выбираемый среди всех рёбер, инцидентных для данной вершины. Вершина ранга $l$ обозначается как $v^{(l)}$, $l \in \{1,2,..., L\}$. Также вводится понятие очага заражения ранга $l$, который обозначается как $\tilde{v}^{(l)}$ и является, по сути, вершиной ранга $l$. 

Одной из самых важных задач данного подхода к моделированию является задача распознавания фрактального графа, при этом различают два вида распознавания: неявное -- определение фракталньости графа и установление некой $n$-вершинной затравки, на которой он базируется; явное -- представление множества рёбер для всякого ранга в явном виде или же представление порождающей траектории в явном виде для некоторого заданного графа $G$. Для этого используется несколько алгоритмов, показанных в  %REF: https://new-disser.ru/_avtoreferats/01005400379.pdf
\cite{Utakaeva_disser:2011} и 
%REF: https://cyberleninka.ru/article/v/raspoznavanie-predfraktalnyh-grafov-s-zatravkoy-udovletvoryayuschey-usloviyu-ore
и \cite{Reznikov:2010} обозначаемых $\alpha_1 ... \alpha_4$ %REF: https://www.dissercat.com/content/issledovanie-svoistv-i-raspoznavanie-predfraktalnykh-grafov/read
\cite{Reznikov_disser:2013}.

Как показано в %REF: https://cyberleninka.ru/article/v/otsenka-diametra-oblasti-rasprostraneniya-virusov-po-modelyam-na-predfraktalnyh-grafah 
\cite{Bajaramukova:2014} и 
%REF: https://new-disser.ru/_avtoreferats/01005400379.pdf
\cite{Utakaeva_disser:2011}, данные модели могут быть использованы для  описания распространения  и протекания как  инфекционных заболеваний, так и для эпидемий <<компьютерных вирусов>>, особенно в локальных сетях, т.к их топография зачастую подобна предфрактальным или фрактальным графам; модели, которые можно построить с помощью данного подхода, могут быть использованы для выявления кластеров заражения или очагов заражения (для эпидемий, не предполагающих иммунитета). Также, значительным преимуществом данных моделей является то, что для распознавания фрактального графа  имеется набор готовых алгоритмов $\alpha_{1...4}$, являющихся полиномиальными; модели позволяют учитывать уровень иммунитета каждого человека или каждого устройства в сети; уже существуют готовые реализации данных моделей в виде программных реализаций. Также, для прикладной реализации важной является   вычислительная сложность, для $\alpha_2 = O(|E|L)$ (если <<старые>> ребра не пересекаются) и  $\alpha_{1,3,4} \leqslant O(|E|L)$, где $|E|$ понимается как количество ребер. 

Недостатком данного подхода можно назвать то, что в силу конечного числа шагов трудно добиться полной фракталньости графа и зачастую используются приближенно фрактальные модели.


\subsection{SIR - SЕIFDR модель}

SIR модель впервые была предложена Кермаком и МакКендриком в 1927 г. для описания заразных заболеваний в закрытых популяциях с течением времени. Модель предполагает фиксированный  размер популяции (без смертей от болезни и естественных причин, без рождаемости), мгновенность инкубационного периода возбудителя болезни, равную продолжительность заразности и самой болезни, полною однородность популяции в терминах возраста, пола, пространства или социальной структуры, т.е данные параметры не учитываются. Модель состоит из системы нелинейных дифференциальных уравнений: \\
\begin{equation} \label{SIR_model:1}
	\dfrac{dS}{dt} =  -\beta SI , 
	\dfrac{dI}{dt} = \beta SI - \gamma I ,
	\dfrac{dR}{dt} = \gamma I , 	
\end{equation}
где $t$ -- время, $S(t)$ -- количество подозреваемых(Suspected), $I(t)$ -- инфицированных(Infected), $R(t)$ -- выздоровевшие с иммунитетом (Recovered), $\beta$ -- интенсивность инфицирования, $ \gamma $ -- интенсивность выздоровления. Важным параметром, управляющим развитием модели, является эпидемиологический порог $R_0 = \dfrac{\beta S}{\gamma} $. Если $R_0 < 1$, то один больной человек заразит  менее одного человека перед своим излечением, т.е вспышка иссякнет; если $R_0 > 1$, то инфекция будет распространяться, т.е $\dfrac{dI}{dt} > 0$  %REF: http://mathworld.wolfram.com/Kermack-McKendrickModel.html
% и прочие ссылки с нижней части этой страницы
\cite{Wolfram_MW:SIR} \cite{Anderson_May:1979}, \cite{Kermack_McKendrick:1927}.
Данная модель и ее частная версия Кермака-МакКендрика настолько интересны, насколько и просты, что делает их подходящими только для описания  коротких вспышек не летальных заболеваний, например, холеры в Лондоне в 1865 г. Для описания других болезней необходимо исследовать их особенности и, в первую очередь, состояния, которые свойственны больным. Так возможно  модифицировать базовую SIR модель до модели, учитывающей смертности и заразность трупов до их захоронения. Такая модель называется <<SEIFDR модель>> и подробна исследована  в %REF: Быкова. Мультиагентный подход
\cite{Bykova:2015} и является глубокой модификацией SIR, где вводятся три новых состояний и ряд переменных, от которых динамика этих состояний зависит. Так, SEIFDR модель описывается системой:
%\begin{equation} \label{SEIFDR_model:1}
		%\dfrac{dS}{dt} = -\beta_I S I + \beta_F S F, 
	%	\dfrac{dE}{dt} = \beta_I S I + \beta_F S F - \alpha E, \\
	%	\dfrac{dI}{dt} = \alpha E - I \gamma_D ( (1 - \delta_D) + \gamma_D \delta_D), \\
	%	\dfrac{dF}{dt} = \gamma_D \delta_D I - \gamma_F F, \\
	%	\dfrac{dR}{dt} = \gamma_I(1-\delta_D) I + \gamma_F F,
%\end{equation}
\begin{gather} 
\nonumber	\dfrac{dS}{dt} = -\beta_I S I + \beta_F S F, \\
\nonumber	\dfrac{dE}{dt} = \beta_I S I + \beta_F S F - \alpha E, \\
	\dfrac{dI}{dt} = \alpha E - I \gamma_D ( (1 - \delta_D) + \gamma_D \delta_D), \label{SEIFDR_model:1} \\
\nonumber	\dfrac{dF}{dt} = \gamma_D \delta_D I - \gamma_F F, \\
\nonumber	\dfrac{dR}{dt} = \gamma_I(1-\delta_D) I + \gamma_F F,
\end{gather}
,  где $\beta_I$ и $\beta_F$ -- интенсивность контактов между людьми и на похоронах, $\alpha$ --  интенсивность инкубационного периода, $\gamma_I$ интенсивность выздоровления, $\gamma_D$  интенсивность умирания агентов, $\gamma_F$ --   интенсивность похорон, $\delta_D$ -- вероятность смертности 
%REF: http://currents.plos.org/outbreaks/article/obk-14-0043-modeling-the-impact-of-interventions-on-an-epidemic-of-ebola-in-sierra-leone-and-liberia/
%REF: http://currents.plos.org/outbreaks/article/modeling-the-impact-of-interventions-on-an-epidemic-of-ebola-in-sierra-leone-and-liberia/
\cite{Plos_Outbreak:1}, \cite{Plos_Outbreak:2}. 
Преимущества: чистый математический подход, выраженный СДУ; наличие большого количества готового имитационного ПО, способного быстро и достаточно точно решать \eqref{SEIFDR_model:1}; высокая прогностическая точность при правильном подборе параметров для заданной болезни; возможность динамически расширять возможные состояния в будущем.

Недостатки: система требует длительной калибровки параметров для максимально точного описания болезни; требуется минимально зашумленная статистика хотя бы инфицированным и умершим для сравнения с работой модели; после калибровки модели для данной болезни на одном городе необходимо тестировать откалиброванную модель на другом городе для избежания создания модели, которая будет только <<подогнанной>> под конкретный случай, но не имеющей прогностических свойств; каждая новая болезнь требует новой калибровки с нуля; модель не учитывает социальную структуру популяции.
\subsection{Выводы о существующих моделях}


Как было показано выше, существует широкое множество различных подходов к имитированию распространения инфекционных заболеваний. При этом каждый из таких подходов обладает своим собственным рядом преимуществ и недостатков. Для реализация и сравнения выбраны две модели: мультиагентная модель, за простоту и скорость, и SEIFDR модель за её строгость и гибкость.