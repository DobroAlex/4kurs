\newpage
\parindent=1cm %красная строка
\begin{center}
		
		\section{Подходы к моделированию процесса распространения инфекционных заболеваний}
		
\end{center}

Эпидемии всегда являлись одной из наиболее острых проблем медицины. Внезапно возникающие эпидемии формируют сложную ситуацию, динамически изменяющуюся во времени. В таких условиях медики и связанные чрезвычайные службы зачастую не могут принять адекватного решения и только порождают дополнительный хаос, неэффективно расходуют ресурсы, не способны в целом положительно повлиять на ход эпидемии или реализовать меры по борьбе с заболеванием. 


Логично предположить, что одним из способов снижения угрозы для населения и уменьшения хаоса среди медиков и прочих ответственных служб являются меры предупредительного характера. Предупредительные меры могут иметь двойственный характер: 
\begin{enumerate}
	\item Во-первых, на основании регулярно обновляемых статистических данных, описывающих эпидемиологическую обстановку, возможно создавать кратковременную прогнозирующую модель, что позволит обнаружить эпидемию ещё до её начала;
	\item Во-вторых, учет данных о прошлых подобных вспышках заболеваемости позволит составить прогноз для ответственных служб, описывающий характер необходимых ресурсов, их количество и область применения, что в свою очередь позволит уменьшить порождаемый вспышкой хаос.
\end{enumerate}

Подобная идея не является революционно новой и  к моделированию распространения заболеваний существует  достаточно много подходов, основанных на разных принципах:полигамные модели,  цепи Маркова, агентный подход, стохастические эксперименты, дифференциальные уравнения на графах, различные алгоритмы на графах. Далее будут подробно рассмотрены некоторые из этих подходов.

Некотрорые общие допущения для всех моделей:
\begin{itemize}
	\item Равномерность и стационарность в смысле неизменяемости во времени  распределения возрастов, то есть все в популяции живут до некоторого возраста $\boldsymbol{L}$ и затем умирают и для каждого возраста (включительно до $\boldsymbol{L}$) количество людей в этой возрастной группе примерно или строго равно. Этот подход хорошо подходит для экономически развитых стран и стран, находящихся на постиндустриальном этапе экономического развития, где детская смертность мала и большинство людей доживают до ожидаемого возраста. Данное предположение может быть отвергнуто для стран, не подходящих под условия выше, например, при моделировании распространения лихорадки Эбола в странах Центральной и Южной Африки;
	\item Гомогенность или однородность перемещений внутри популяции, то есть индивидуумы  в популяции under scrutiny assort %TODO: перевести
	и контактируют случайно, не замыкайся в более мелких подгруппах. Данное допущение является спорным, т.к социальная структура широкомасштабная и сложная, то есть индивидуумы внутри одной большой группы могут находиться в таких социальных отношениях, что большинство контактов будет приходиться на их собственную подгруппу, в то время как количество контактов вне группы будет предельно малым. Однако, такое допущение имеет место быть для упрощения построения модели   и понимания результатов моделирования.
\end{itemize}
