\newpage
\parindent=1cm %красная строка
\begin{center}
		
		\section{Подходы к моделированию процесса распространения инфекционных заболеваний}
		
\end{center}

Эпидемии всегда являлись одной из наиболее острых проблем медицины. Внезапно возникающие эпидемии формируют сложную ситуацию, динамически изменяющуюся во времени. В таких условиях медики и связанные чрезвычайные службы зачастую не могут принять адекватного решения и только порождают дополнительный хаос, неэффективно расходуют ресурсы, не способны в целом положительно повлиять на ход эпидемии или реализовать меры по борьбе с заболеванием. 


Логично предположить, что одним из способов снижения угрозы для населения и уменьшения хаоса среди медиков и прочих ответственных служб являются меры предупредительного характера. Предупредительные меры могут иметь двойственный характер: 
\begin{enumerate}
	\item Во-первых, на основании регулярно обновляемых статистических данных, описывающих эпидемиологическую обстановку, возможно создавать кратковременную прогнозирующую модель, что позволит обнаружить эпидемию ещё до её начала;
	\item Во-вторых, учет данных о прошлых подобных вспышках заболеваемости позволит составить прогноз для ответственных служб, описывающий характер необходимых ресурсов, их количество и область применения, что в свою очередь позволит уменьшить порождаемый вспышкой хаос.
\end{enumerate}

Подобная идея не является революционно новой и  к моделированию распространения заболеваний существует  достаточно много подходов, основанных на разных принципах:полигамные модели,  цепи Маркова, агентный подход, стохастические эксперименты, дифференциальные уравнения на графах, различные алгоритмы на графах. Далее будут подробно рассмотрены некоторые из этих подходов.

Некоторые общие допущения для всех моделей:
\begin{itemize}
	\item Равномерность и стационарность в смысле неизменяемости во времени  распределения возрастов, то есть все в популяции живут до некоторого возраста $\mathit{L}$ и затем умирают и для каждого возраста (включительно до $\mathit{L}$) количество людей в этой возрастной группе примерно или строго равно. Этот подход хорошо подходит для экономически развитых стран и стран, находящихся на постиндустриальном этапе экономического развития, где детская смертность мала и большинство людей доживают до ожидаемого возраста. Данное предположение может быть отвергнуто для стран, не подходящих под условия выше, например, при моделировании распространения лихорадки Эбола в странах Центральной и Южной Африки;
	\item Гомогенность или однородность перемещений внутри популяции, то есть индивидуумы  в популяции under scrutiny assort %TODO: перевести
	и контактируют случайно, не замыкайся в более мелких подгруппах. Данное допущение является спорным, т.к социальная структура широкомасштабная и сложная, то есть индивидуумы внутри одной большой группы могут находиться в таких социальных отношениях, что большинство контактов будет приходиться на их собственную подгруппу, в то время как количество контактов вне группы будет предельно малым. Однако, такое допущение имеет место быть для упрощения построения модели   и понимания результатов моделирования.
\end{itemize}


Также опишем один общий принцип разделения моделей по признаку случайности. По такому признаку  модели можно разделить на стохастические и детерминистические. Стохастические модели предполагают наличие случайных величин  и являются инструментом для оценки распределения вероятности потенциальных исходов  посредством допущения случайных изменений в одной или нескольких переменных с ходом времени. Такие модели зависят от случайных изменений рисков, связанных с продолжительностью экспозиции, вероятности заболевания и прочих динамических изменяющихся параметров заболевания. Для описания различных этапов заболевания часто используют буквы M(maternally-derived immunity), S(suspicious), E(exposed), I(infected), R(recovered).

\subsection{Агентный подход к имитационному моделированию процесса распространения заболевания}

Существует несколько понятий определения "агент" и "агентный подход" в целом. Обобщая их, можно выделить, что  \textit{агент}  -- это некоторая сущность, имеющая активность, автономное поведение, способная самостоятельно принимать решения в соответствии с некоторой совокупностью правил, взаимодействовать с окружающей средой, если такая предусмотрена, и другими агентами, если такие существуют. Основное поле применения таких моделей -- децентрализованные систем, динамика функционирования которых не является следствием некоторых, зачастую внешних, правил и законов, но наоборот, такие правила и законы являются внутренним неотъемлемым результатом работы множества агентов. Такие модели обычно являются  дискретно-событийные или дискретные с непрерывными элементами, то есть гибридные. 

Данные подход удобно иллюстрировать на примере моделирования распространения гриппа (обычно, гриппа А), т.к грипп вносит весомый вклад в смертность при инфекционных заболеваниях и при этом хорошо известны его способы передачи и формы течения болезни. Также по данному заболеванию собрана достаточно обширная статистика, который можно использовать для проверки точности модели и последующей калибровки.  Такая модель достаточно подробно рассмотрена в работе М.А. Кондратьева, Р.И. Ивановского, Л.М. Цымбалова <<Применение агентного подхода к имитационному моделированию процесса распространения заболевания>>. В данной работе ставится цель  разработки модели, способной определять число больных гриппом в каждый день определенного краткосрочного периода (2-3 недели) во время сезонных вспышек, то есть построение краткосрочного прогноза. Для этого популяция  разбивается на возрастные группы, позволяющие характеризовать контактов с другими людьми в сутки и, что важно, потенциально посещаемые места. В работе вводятся два основных типа объектов: локации и агенты и связанные с ними события перемещения агентов между локациями, протекания заболевания у агентов и событие <<контакта>>. Также в данной работе вводятся правила, описывающие поведение агентов с помощью UML Statecharts (далее <<стейтчарты>>). С помощью таких стейтчартов описывается переход между различными состояниями индивидуума как вне течения болезни, так и после, и может быть обозначен с помощью графической нотации как: 
\newline  
%TODO: fix that shit 
X  $  \xrightarrow{\text{приобретает возможность заражать}} $ \newline Y    $ \xrightarrow{\text{переход в состояние инфицированности и полноценный ход болезни}} $   A   $ \xrightarrow{\text{снижение заразности}} $   B   $ \xrightarrow{\text{еще большее снижение заразности}} $  \newline C   $ \xrightarrow{\text{переход в состояние полного выздоровления с возможностью повторного заражения через некотрое время}} $   S 
\newline
где X -- этап инкубационного периода, когда агент еще не заразен, Y -- также этап инкубационного периода, где агент становится заразным, A -- начало этапа болезни с явным выражением симптомов, этапы B и C представляют собой постепенный спад заразности и после C наступает выздоровление, когда индивидуум более не заразен, однако снова может оказаться больным с течением времени, если не выработался иммунитет. 


Данная модель обладает рядом преимуществ: высокая скорость разработки в силу выбора готового имитационного ПО (специфично для данной рассматриваемой работы, хотя в целом верно для разработки любой агентной модели в силу простоты инкапсуляции агентов и построения связей между объектами); высокая скорость работы, так как выбранная модель допускает высокий параллелизм в рамках одной итерации, когда вычисления для каждого объекта-места допускают распараллеливание; свобода от дифференциальных уравнений значительно упрощает расчеты, в том числе исчезает необходимость использовать численные методы и связанные с ними понятия <<сходимости>> и <<устойчивости>>, при этом последнее особенно важно, так как входные статистические данные зачастую имеют шумы;  сам агентный подход допускает быструю модификацию модели и возможность учета применяемых административных мер  и их влияния на ход распространения инфекции.

Недостатки: в силу собственной стохастичности, модель требует многократного запуска  и оценкой экспертом полученных результатов для получения некоторого <<среднего результата>>; достаточно сложная модель может потребовать много времени на разработку даже при использовании готового имитационного ПО, при этом такая модель может оказаться ресурсозатратной в терминах машинных мощностей (время, память); невозможно точно определить, насколько точна данная модель по отношению к реальным, кроме как посредством многократного сравнения результатов работы модели и собранных существующих данных, при этом невозможно убедиться, что данная модель и текущие параметры могут быть применимы для прогнозирования, а не только откалиброваны для соответствия предыдущим результатам; стохастическая природа модели при компьютерной реализации опирается на генерацию случайных чисел, что требует использования мощных и проверенных генераторов. 

\subsection{Применение случайных графов для распознавания и анализа инфекционных заболеваний}

В качестве другого подхода к моделированию распространения инфекционных заболеваний можно рассмотреть переносчиков заболеваний как  саморазмножающиеся сущности, т.е такие сущности, которые способны самореплицироваться, при этом свойство  саморепликации передается не только между <<родителями>> и <<потомками>> (вертикальный перенос), но возможен и горизонтальный перенос, т.е передача свойства саморепликации и, возможно, но не обязательно, других свойств, объектам, которые так или иначе соседние или контактируют с данным.  Примерами являются размножение компьютерных вирусов и сетевых червей  %REF: [1] с https://cyberleninka.ru/article/v/sluchaynye-grafy-kak-model-sredy-rasprostraneniya-i-vzaimodeystviya-samorazmnozhayuschihsya-obektov
 или инфицирование организмов вирусами и инфекциями %REF: [3;5] оттуда же 
. 
В данных моделях обычно применяются графы, используемые для моделирования эпидемий <<мобильных червей>> (т.е вредоносных программ, распространяющихся между устройствами при помощи беспроводных протоколов типа Bluetooth или Wi-Fi, широко применяемых в мобильных устройствах, отсюда и происходит название) %REF: http://www.ssc.smr.ru/media/journals/izvestia/2016/2016_4_744_748.pdf стр 744
%REF:  КЛИМЕНТЬЕВ К.Е. Компьютерные вирусы и антивирусы: взгляд программиста Москва: ДМК-Пресс, 2013. 656с. стр 27
или воздушно-капельных инфекций среди высших  животных.   %REF: [1] https://cyberleninka.ru/article/v/sluchaynye-grafy-kak-model-sredy-rasprostraneniya-i-vzaimodeystviya-samorazmnozhayuschihsya-obektov
Согласно К.Е. Климентьеву, %REF: https://cyberleninka.ru/article/v/sluchaynye-grafy-kak-model-sredy-rasprostraneniya-i-vzaimodeystviya-samorazmnozhayuschihsya-obektov
для подобных эпидемий характерны следующие черты:
\begin{itemize}
	\item Постоянные изменения топологии среды моделирования в связи с высокой мобильностью  объектов;
	\item ограниченный радиус <<заражения>>, как для биологических агентов, так и для компьютерных вирусов, обусловленный физическими свойствами оных.
\end{itemize}


Для моделирования среды существования таких объектов используется класс сетей, называемый <<специальным>> (<<Ad hoc>>), представляющий собой множество случайных графов с разнообразной топологией.  