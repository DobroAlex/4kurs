\newpage
\parindent=1cm %красная строка? 
\begin{center}
	\addcontentsline{toc}{section}{Введение} %Убираем номер , даём имя в оглавлении 
	\section*{Введение} %сам текст заголовка 
	\pagestyle{plain} % нумерация выкл.
	\setcounter{page}{3} % начать нумерацию с номера три
\end{center}


Несмотря на  значительные достижения в области борьбы с инфекционными заболеваниям и порождаемыми ими эпидемиями, данная проблема и связанные с ней задачи все ещё актуальны. Одной из задач, возникающих в связи с борьбой с инфекционными заболеваниями, является исследование путей и способов передачи инфекций. В данной области возникает задача прогнозирования распространения конкретного инфекционного заболевания в некоторой заданной среде, то есть задача построения математической модели, описывающей скорость и масштабы распространения заболевания и его последствия: количество инфицированных и умерших, пространственные масштабы заражения, затраты средств и ресурсов,  необходимые для изоляции больных и последующего излечения. 

Математическое моделирование является мощным и гибким инструментом для исследования реальных процессов и объектов, а также связей между ними. Математическое моделирование удобно применять в тех ситуациях, когда проведение эксперимента с реальными субъектами затруднено по любой причине, что делает его подходящим способом исследования распространения инфекционных заболеваний. Заметим, что учитывая специфическую природу заболеваний и путей их передачи, которые будут изучены и исследованы далее,  важным фактором для проверки адекватности построенной модели является наличие статистической информации, позволяющей тестировать соответствие модели реальным прецедентам.

Важно заметить, что модели, разработанные в данной области, не являются идеальными и универсальными. Основные допущения, принимаемые при построении модели, и недостатки этих моделей будут подробно рассмотрены далее. 

Целью данной работы является анализ существующих моделей распространения инфекционных заболеваний и  реализация нескольких различных  моделей, позволяющих прогнозировать распространение конкретных заболеваний на основе существующих статистических данных, реализовать полноценное приложение для использования \newline специалистами-эпидемиологами.  Для этого необходимо решить нижеизложенный комплекс задач:
\begin{itemize}
	\item Проанализировать существующие подходы к моделированию, изучить сильные и слабые стороны этих подходов, принимаемые допущения. На основании этого анализа выбрать подход, который будет использоваться при построении собственной модели;
	
	\item выбрать заболевание для тестирования, подробно изучить его и собрать статистические данные о нем, найти статистику, позволяющую сравнить результаты моделирования с реальными;
	
	\item выбрать среду и средства	 моделирования;
	
	\item разработать и реализовать базовую модель;
	
	\item на основе базовой модели создать модель повышенной точности и детализации;
	
	\item итеративно тестировать и улучшать результаты модели повышенной точности; 
	
	\item разработать отдельное приложение для ввода данных в модель и визуализации полученных моделированием результатов для конечного пользователя -- специалиста-медика или эпидемиолога. 	
\end{itemize}

Для решения поставленного комплекса задач использовались методы математической статистики и теории вероятности, дискретной математики, математического анализа. Разработанные модели основываются на методах агентного / мультиагентного моделирования, их реализации опираются на объектно-ориентированное программирование и современные средства  программного анализа и визуализации данных. 

Объект исследования: изучение распространения инфекционных заболеваний и соответствующих прогностических моделей.

Предмет исследования: построение и анализ моделей распространения инфекционных заболеваний.

Практическая ценность результатов работы: проведено исследование существующих подходов к моделированию,	разработана модель  и программный продукт, реализующий эту модель, также разработан программный продукт для ввода данных в модель специалистом и динамической визуализации результатов моделирования. Разработанные приложения позволяют дать количественный и качественный прогноз распространения инфекционного заболевания для любых  достаточно точно и полно описанных популяции и заболевания.

В первой главе данной выпускной квалификационной работы рассматриваются и анализируются существующие подходы к моделированию распространения инфекционных заболеваний: агентный и мультиагентный подход, алгоритмы на основе случайных  и предфрактальных графов, детерминистический подход на основе дифференциальных уравнений; подводятся итоги и проводится сравнение с выбором наиболее перспективных подходов.

Во второй главе обосновывается выбор целевого заболевания для исследования, приводится подробное описание тех его особенностей, которые необходимы для построения моделей; проводится поиск статистики; проводится анализ средств моделирования с выбором оптимальных; описывается построение моделей, проводившиеся эксперименты, процесс повышения точности модели  и сравнение результатов работы моделей.
