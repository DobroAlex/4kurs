\documentclass[a4paper,14pt,russian]{extreport}	%A4 бумага, 14 кегль, русский язык 
\usepackage{extsizes}
\usepackage[onehalfspacing]{setspace} % поулторный интервал %https://proft.me/2013/06/9/latex-ukazanie-mezhstrochnogo-intervala

\usepackage{cmap} % для кодировки шрифтов в pdf
\usepackage[T2A]{fontenc}
%\usepackage{pscyr}
%\usepackage{graphicx} % для вставки картинок
\usepackage{mathptmx} %поддержка textbf
\usepackage{makecell}
\usepackage{textcomp}
\usepackage{multirow} % улучшенное форматирование таблиц
\usepackage{ulem} % подчеркивания

%полужирный шрифт http://tostudents.ru/2009/12/08/smena-shriftov-v-latex-tekst-i-formuly/
\renewcommand{\rmdefault}{ftm} % Times New Roman
\usepackage[utf8]{inputenc}%включаем свою кодировку: koi8-r или utf8 в UNIX, cp1251 в Windows
%\usepackage[]{babel}	%больше поддержки русского языка 
\usepackage[english,russian, russianb]{babel}%используем русский и английский языки с переносами
\usepackage{amssymb,amsfonts,amsmath,mathtext,cite,enumerate,float} %подключаем нужные пакеты расширений

\usepackage[pdftex]{graphicx} %хотим вставлять в диплом рисунки?
\usepackage{cmap} % Улучшенный поиск русских слов в полученном pdf-файле
%\graphicspath{{images/}}%путь к рисункам
\usepackage{fancyhdr}%оформление нумерации 
\usepackage{tableof} %поддержка табличек
\usepackage{mathptmx}%
\usepackage{anyfontsize}% http://texblog.org/2012/08/29/changing-the-font-size-in-latex/
\usepackage{t1enc}%
\usepackage{cite}
\usepackage{graphicx}
\usepackage{xcolor} % цвет текста или фона https://tex.stackexchange.com/questions/136742/changing-background-color-of-text-in-latex
\graphicspath{{Images/}} %http://dkhramov.dp.ua/Comp.TexIncludeGraphics#.WwLdJa3sTMU
\DeclareGraphicsExtensions{.pdf,.png,.jpg} %http://dkhramov.dp.ua/Comp.TexIncludeGraphics#.WwLdJa3sTMU
 %https://tex.stackexchange.com/questions/17734/cannot-determine-size-of-graphic
\makeatletter
\renewcommand{\@biblabel}[1]{#1.} % Заменяем библиографию с квадратных скобок на точку:
\makeatother

\usepackage{geometry} % Меняем поля страницы
%QUEST: 3cm или 2 ? Ес\usepackage{cmap} % Улучшенный поиск русских слов в полученном pdf-файлели 3, придется менять форматирование заголовка 
\geometry{left=3cm}% левое поле
\geometry{right=2cm}% правое поле
\geometry{top=2cm}% верхнее поле
\geometry{bottom=2cm}% нижнее поле


%Рисование флоучартов
%https://www.overleaf.com/learn/latex/LaTeX_Graphics_using_TikZ:_A_Tutorial_for_Beginners_(Part_3)%E2%80%94Creating_Flowcharts
%TODO: доделать
\usepackage{tikz} 
\usetikzlibrary{arrows, shapes.geometric}

\tikzstyle{SEIFDRState} = [rectangle, rounded corners, minimum width=3cm, minimum height=1cm,text centered, draw=black]
\tikzstyle{SEIFDRInnerState} =  [trapezium, trapezium left angle=70, trapezium right angle=110, minimum width=1cm, minimum height=1cm, text centered, draw=black]


\tikzstyle{arrow} = [thick,->,>=stealth]

\renewcommand{\theenumi}{\arabic{enumi}}% Меняем везде перечисления на цифра.цифра
\renewcommand{\labelenumi}{\arabic{enumi}}% Меняем везде перечисления на цифра.цифра
\renewcommand{\theenumii}{.\arabic{enumii}}% Меняем везде перечисления на цифра.цифра
\renewcommand{\labelenumii}{\arabic{enumi}.\arabic{en\part{title}umii}.}% Меняем везде перечисления на цифра.цифра
\renewcommand{\theenumiii}{.\arabic{enumiii}}% Меняем везде перечисления на цифра.цифра
\renewcommand{\labelenumiii}{\arabic{enumi}.\arabic{enumii}.\arabic{enumiii}.}% Меняем везде перечисления на цифра.цифра
\addto\captionsrussian{\def\refname{Список используемой литературы}}
\renewcommand{\rmdefault}{ftm}
%NB: три команды ниже переопределяют некотрые шрифты  и дают поддержку жирного и прочиах текстов https://www.linux.org.ru/forum/general/4219163
\renewcommand{\rmdefault}{cmr} % Шрифт с засечками
\renewcommand{\sfdefault}{cmss} % Шрифт без засечек
\renewcommand{\ttdefault}{cmtt} % Моноширинный шрифт
\renewcommand*\thesection{\arabic{section}}
    
\begin{document}
	\numberwithin{equation}{subsection} %Стилистика нумерования уравненений в \begin{equation} \label{name:\d}. Тут -- номер главы.сабсекция.номер_уравнения_с_лэйблом_name
	% https://en.wikibooks.org/wiki/LaTeX/Advanced_Mathematics#Equation_numbering
	\renewcommand{\bibname}{Список использованной литературы}
	%\pagestyle{empty} % нумерация выкл.
	    \begin{titlepage}
    \newpage
	\pagestyle{empty} % нумерация выкл.
    \begin{center}
    
	
	{\fontsize{13}{15.6}\selectfont МИНИСТЕРСТВО ОБРАЗОВАНИЯ И НАУКИ РОССИЙСКОЙ ФЕДЕРАЦИИ}\\ 
    \normalsize  {Федеральное государственное автономное образовательное учреждение высшего образования} \\
    
    \large \textbf{<<Крымский  федеральный  университет имени В. И. Вернадского>>} \\  \vspace{2mm}
    (ФГАОУ ВО «КФУ им. В. И. Вернадского»)\\
    
    \textbf{Таврическая академия (структурное подразделение ) \\
    \vspace{2mm}
    Факультет математики и информатики} \\
    \vspace{2mm}
    Кафедра прикладной математики 
    \end{center}
    \vspace{1em}

    \begin{center}
	\normalsize Консманов Алексей Витальевич \\
    \LARGE \textbf{Моделирование распространения инфекционных заболеваний} \\
    \vspace{1em}
    \normalsize Выпускная квалификационная работа 
    \end{center}

    \vspace{1em}
    Обучающегося \hspace*{3cm} \underline{4} курса 
    
    
    Направления подготовки\hspace*{8mm} \underline{01.03.04. Прикладная математика}
    
        
    Форма обучения\hspace*{2.75cm} \underline{очная}\\
    
    
    Научный руководитель
    
    доцент кафедры прикладной математики, 
    
    кандидат физико-математических наук \hspace*{1cm} Ю.Ю. Дюличева
    %\begin{center}
    %	\begin{tabbing}	%http://www.intuit.ru/studies/courses/1137/137/lecture/3835%3Fpage%3D5
    %		\hspace{3cm}Обучающегося \hspace{3cm} \textbf{3 курса}\\ %Быдлокод?
    %		\hspace{2.7cm}Направления подготовки \hspace*{5mm}  \textbf{01.03.04}\\
    %		\hspace{3cm}Форма обучения \hspace{26mm} \textbf{очная}
    %	\end{tabbing}
    
%	\vspace {3em}
%    \flushleft Научный руководитель \hspace{20mm}  старший преподаватель 
    
%    \hspace{75mm}кафедры прикладной математики  
    
    
%    \hspace{75mm}В. А. Лушников
%	\end{center}
    \vspace{\fill}

    \begin{center}
    Симферополь 2019
    \end{center}

    \end{titlepage}% это титульный лист

	\tableofcontents % это оглавление, которое генерируется автоматически
	\thispagestyle{empty}%отключает нумерование страниц до введения включительно 
	%\addcontentsline{toc}{section}{Введение}% будет костыльно выглядеть
	\newpage
\parindent=1cm %красная строка? 
\begin{center}
	\addcontentsline{toc}{section}{Введение} %Убираем номер , даём имя в оглавлении 
	\section*{Введение} %сам текст заголовка 
	\pagestyle{plain} % нумерация выкл.
	\setcounter{page}{3} % начать нумерацию с номера три
\end{center}


Несмотря на  значительные достижения в области борьбы с инфекционными заболеваниям и порождаемыми ими эпидемиями, данная проблема и связанные с ней задачи все ещё актуальны. Одной из задач, возникающих в связи с борьбой с инфекционными заболеваниями, является исследование путей и способов передачи инфекций. В данной области возникает задача прогнозирования распространения конкретного инфекционного заболевания в некоторой заданной среде, то есть задача построения математической модели, описывающей скорость и масштабы распространения заболевания и его последствия: количество инфицированных и умерших, пространственные масштабы заражения, затраты средств и ресурсов,  необходимые для изоляции больных и последующего излечения. 

Математическое моделирование является мощным и гибким инструментом для исследования реальных процессов и объектов, а также связей между ними. Математическое моделирование удобно применять в тех ситуациях, когда проведение эксперимента с реальными субъектами затруднено по любой причине, что делает его подходящим способом исследования распространения инфекционных заболеваний. Заметим, что учитывая специфическую природу заболеваний и путей их передачи, которые будут изучены и исследованы далее,  важным фактором для проверки адекватности построенной модели является наличие статистической информации, позволяющей тестировать соответствие модели реальным прецедентам.

Важно заметить, что модели, разработанные в данной области, не являются идеальными и универсальными. Основные допущения, принимаемые при построении модели, и недостатки этих моделей будут подробно рассмотрены далее. 

Целью данной работы является анализ существующих моделей распространения инфекционных заболеваний и  создание нескольких различных собственных моделей, позволяющих прогнозировать распространение конкретных заболеваний на основе существующих статистических данных, реализовать полноценное приложение для использования \newline специалистами-эпидемиологами.  Для этого необходимо решить нижеизложенный комплекс задач:
\begin{itemize}
	\item Проанализировать существующие подходы к моделированию, изучить сильные и слабы стороны этих подходов, принимаемые допущения. На основании этого анализа выбрать подход, который будет использоваться при построении собственной модели.
	
	\item выбрать заболевание для тестирования, подробно изучить его и собрать статистические данные о нем, найти статистику, позволяющую сравнить результаты моделирования с реальными;
	
	\item выбрать среду моделирования;
	
	\item разработать и реализовать базовую модель;
	
	\item на основе базовой модели создать модель повышенной точности и детализации;
	
	\item итеративно тестировать и улучшать результаты модели повышенной точности; 
	
	\item разработать отдельное приложение для ввода данных в модель и визуализации полученных моделированием результатов для конечного пользователя -- специалиста-медика или эпидемиолога. 	
\end{itemize}

Для решения поставленного комплекса задач использовались методы математической статистики и теории вероятности, дискретной математики, математического анализа. Разработанные модели основываются на методах агентного / мультиагентного моделирования, их реализации опираются на объектно-ориентированное программирование и современные средства  программного анализа и визуализации данных. 

Объект исследования: изучение распространения инфекционных заболеваний и соответствующих прогностических моделей.

Предмет исследования: построение и анализ моделей распространения инфекционных заболеваний.

Практическая ценность результатов работы: проведено исследование существующих подходов к моделированию,	разработана модель  и программный продукт, реализующий эту модель, также разработан программный продукт для ввода данных в модель специалистом и динамической визуализации результатов моделирования. Разработанные приложения позволяют дать количественный и качественный прогноз распространения инфекционного заболевания для любых  достаточно точно и полно описанных популяции и заболевания.

В первой главе данной выпускной квалификационной работы рассматриваются и анализируются существующие подходы к моделированию распространения инфекционных заболеваний: агентный и мультиагентный подход, алгоритмы на основе случайных  и предфрактальных графов, детерминистический подход на основе дифференциальных уравнений; подводятся итоги и проводится сравнение с выбором наиболее перспективных подходов.

Во второй главе обосновывается выбор целевого заболевания для исследования, приводится подробное описание тех его особенностей, которые необходимы для построения моделей; проводится поиск статистики; проводится анализ средств моделирования с выбором оптимальных; описывается построение моделей, проводившиеся эксперименты, процесс повышения точности модели  и сравнение результатов работы моделей.
 %введение
	\newpage
\parindent=1cm %красная строка
\begin{center}
		
		\section{Подходы к моделированию процесса распространения инфекционных заболеваний}
		
\end{center}

Эпидемии всегда являлись одной из наиболее острых проблем медицины. Внезапно возникающие эпидемии формируют сложную ситуацию, динамически изменяющуюся во времени. В таких условиях медики и связанные чрезвычайные службы зачастую не могут принять адекватного решения и только порождают дополнительный хаос, неэффективно расходуют ресурсы, не способны в целом положительно повлиять на ход эпидемии или реализовать меры по борьбе с заболеванием. 


Логично предположить, что одним из способов снижения угрозы для населения и уменьшения хаоса среди медиков и прочих ответственных служб являются меры предупредительного характера. Предупредительные меры могут иметь двойственный характер: 
\begin{enumerate}
	\item На основании регулярно обновляемых статистических данных, описывающих эпидемиологическую обстановку, возможно создавать кратковременную прогнозирующую модель, что позволит обнаружить эпидемию ещё до её начала;
	\item Учет данных о прошлых подобных вспышках заболеваемости позволит составить прогноз для ответственных служб, описывающий характер необходимых ресурсов, их количество и область применения, что в свою очередь позволит уменьшить порождаемый вспышкой хаос.
\end{enumerate}

Подобная идея не является революционно новой и  к моделированию распространения заболеваний уже  существует  достаточно много подходов, основанных на разных принципах:цепи Маркова, агентный подход, стохастические эксперименты, дифференциальные уравнения на графах и другие  различные алгоритмы на графах, полигамные модели. Далее будут подробно рассмотрены некоторые из этих подходов.

Некоторые общие допущения для всех моделей:
\begin{itemize}
	\item Равномерность и стационарность в смысле неизменяемости во времени  распределения возрастов, то есть все в популяции живут до некоторого возраста $\mathit{L}$ и затем умирают и для каждого возраста (включительно до $\mathit{L}$) количество людей в этой возрастной группе примерно или строго равно. Этот подход хорошо подходит для экономически развитых стран и стран, находящихся на постиндустриальном этапе экономического развития, где детская смертность мала и большинство людей доживают до ожидаемого возраста. Данное предположение может быть отвергнуто для стран, не подходящих под условия выше, например, при моделировании распространения лихорадки Эбола в странах Центральной и Южной Африки;
	\item Гомогенность или однородность перемещений внутри популяции, то есть индивидуумы  в популяции under scrutiny assort %TODO: перевести
	и контактируют случайно, не замыкайся в более мелких подгруппах. Данное допущение является спорным, т.к социальная структура широкомасштабная и сложная, то есть индивидуумы внутри одной большой группы могут находиться в таких социальных отношениях, что большинство контактов будет приходиться на их собственную подгруппу, в то время как количество контактов вне группы будет предельно малым. Однако, такое допущение имеет место быть для упрощения построения модели   и понимания результатов моделирования.
\end{itemize}


Также опишем один общий принцип разделения моделей по признаку случайности. По такому признаку  модели можно разделить на стохастические и детерминистические. Стохастические модели предполагают наличие случайных величин  и являются инструментом для оценки распределения вероятности потенциальных исходов  посредством допущения случайных изменений в одной или нескольких переменных с ходом времени. Такие модели зависят от случайных изменений рисков, связанных с продолжительностью экспозиции, вероятности заболевания и прочих динамических изменяющихся параметров заболевания. Для описания различных этапов заболевания часто используют буквы M(maternally-derived immunity), S(suspicious), E(exposed), I(infected), R(recovered).

\subsection{Агентный подход к имитационному моделированию процесса распространения заболевания}

Существует несколько понятий определения "агент" и "агентный подход" в целом. Обобщая их, можно выделить, что  \textit{агент}  -- это некоторая сущность, имеющая активность, автономное поведение, способная самостоятельно принимать решения в соответствии с некоторой совокупностью правил, взаимодействовать с окружающей средой, если такая предусмотрена, и другими агентами, если такие существуют. Основное поле применения таких моделей -- децентрализованные систем, динамика функционирования которых не является следствием некоторых, зачастую внешних, правил и законов, но наоборот, такие правила и законы являются внутренним неотъемлемым результатом работы множества агентов. Такие модели обычно являются  дискретно-событийными или дискретными с непрерывными элементами, то есть гибридными. 

Данный подход удобно иллюстрировать на примере моделирования распространения гриппа (обычно, гриппа А), т.к грипп вносит весомый вклад в смертность при инфекционных заболеваниях и при этом хорошо известны его способы передачи и формы течения болезни. Также по данному заболеванию собрана достаточно обширная статистика, который можно использовать для проверки точности модели и последующей калибровки.  Такая модель достаточно подробно рассмотрена в  %REF:М.А. Кондратьева, Р.И. Ивановского, Л.М. Цымбалова <<Применение агентного подхода к имитационному моделированию процесса распространения заболевания>>. 
%REF:М.А. Кондратьева, Р.И. Ивановского, Л.М. Цымбалова <<Применение агентного подхода к имитационному моделированию процесса распространения заболевания>>,
описывает модель, способную определять число больных гриппом в каждый день определенного краткосрочного периода (2-3 недели) во время сезонных вспышек, то есть построение краткосрочного прогноза. Для этого популяция  разбивается на возрастные группы, позволяющие характеризовать контактов с другими людьми в сутки и, что важно, потенциально посещаемые места. В работе вводятся два основных типа объектов: локации и агенты и связанные с ними события перемещения агентов между локациями, протекания заболевания у агентов и событие <<контакта>>. Также в данной работе вводятся правила, описывающие поведение агентов с помощью UML Statecharts (далее <<стейтчарты>>). С помощью таких стейтчартов описывается переход между различными состояниями индивидуума как вне течения болезни, так и после, и может быть обозначен с помощью графической нотации как: 

\begin{tikzpicture} {node distance=2cm}
	\node (Suspected) [SEIFDRState] {Suspected};
	

	\node (Exposed) [SEIFDRState, below of=Suspected, yshift=-2cm] {Exposed};
	
	\node (stateX) [SEIFDRInnerState, right of=Exposed, xshift=5.5cm] {X};
	
	\node (stateY) [SEIFDRInnerState, right of=stateX, xshift=5cm] {Y};
	
	\node (Infectious) [SEIFDRState, below of=Exposed, yshift=-2cm] {Infectious};
	
	\node (stateA) [SEIFDRInnerState, right of=Infectious, xshift=1.5cm] {A};
	
	\node (stateB) [SEIFDRInnerState, right of=stateA, xshift=0.75cm] {B};
	
	\node (stateC) [SEIFDRInnerState, right of=stateB, xshift=7.5cm] {C};
	
	\node (Recovered) [SEIFDRState, below of=Infectious, yshift=-2cm] {Recovered};
	
	\draw [arrow] (Suspected) -- node[anchor=south] {Контакт с инфицированным} (Exposed);
	
	\draw [arrow] (Exposed) -- node[anchor=south] {Больной не заразен} (stateX);
	
	\draw [arrow] (stateX) -- node[anchor=south] {Больной заразен} (stateY);
	
	\draw [arrow] (stateY) -- node[anchor=south]  {Проявление симптомов} (Infectious);
	
	\draw [arrow] (Infectious) --  (stateA);
	
	\draw [arrow] (stateA) --  (stateB);
	
	\draw [arrow] (stateB) -- node[anchor=south] {Снижение заразности от А к С} (stateC);

	\draw [arrow] (stateC) -- node[anchor=south] {Выздоровление и приобретение иммунитета} (Recovered);	
\end{tikzpicture}


	



Данная модель обладает рядом преимуществ: высокая скорость разработки в силу выбора готового имитационного ПО (специфично для данной рассматриваемой работы, хотя в целом верно для разработки любой агентной модели в силу простоты инкапсуляции агентов и построения связей между объектами); высокая скорость работы, так как выбранная модель допускает высокий параллелизм в рамках одной итерации, когда вычисления для каждого объекта-места допускают распараллеливание; свобода от дифференциальных уравнений значительно упрощает расчеты, в том числе исчезает необходимость использовать численные методы и связанные с ними понятия <<сходимости>> и <<устойчивости>>, при этом последнее особенно важно, так как входные статистические данные зачастую имеют шумы;  сам агентный подход допускает быструю модификацию модели и возможность учета применяемых административных мер  и их влияния на ход распространения инфекции.

Недостатки: в силу собственной стохастичности, модель требует многократного запуска  и оценкой экспертом полученных результатов для получения некоторого <<среднего результата>>; достаточно сложная модель может потребовать много времени на разработку даже при использовании готового имитационного ПО, при этом такая модель может оказаться ресурсозатратной в терминах машинных мощностей (время, память); невозможно точно определить, насколько точна данная модель по отношению к реальным, кроме как посредством многократного сравнения результатов работы модели и собранных существующих данных, при этом невозможно убедиться, что данная модель и текущие параметры могут быть применимы для прогнозирования, а не только откалиброваны для соответствия предыдущим результатам; стохастическая природа модели при компьютерной реализации опирается на генерацию случайных чисел, что требует использования мощных и проверенных генераторов. 

\subsection{Применение случайных графов для распознавания и анализа инфекционных заболеваний}

В качестве другого подхода к моделированию распространения инфекционных заболеваний можно рассмотреть переносчиков заболеваний как  саморазмножающиеся сущности, т.е такие сущности, которые способны самореплицироваться, при этом свойство  саморепликации передается не только между <<родителями>> и <<потомками>> (вертикальный перенос), но возможен и горизонтальный перенос, т.е передача свойства саморепликации и, возможно, но не обязательно, других свойств, объектам, которые так или иначе соседние или контактируют с данным.  Примерами являются размножение компьютерных вирусов и сетевых червей  %REF: [1] с https://cyberleninka.ru/article/v/sluchaynye-grafy-kak-model-sredy-rasprostraneniya-i-vzaimodeystviya-samorazmnozhayuschihsya-obektov
 или инфицирование организмов вирусами и инфекциями %REF: [3;5] оттуда же 
. 
В данных моделях обычно применяются графы, используемые для моделирования эпидемий <<мобильных червей>> (т.е вредоносных программ, распространяющихся между устройствами при помощи беспроводных протоколов типа Bluetooth или Wi-Fi, широко применяемых в мобильных устройствах, отсюда и происходит название) %REF: http://www.ssc.smr.ru/media/journals/izvestia/2016/2016_4_744_748.pdf стр 744
%REF:  КЛИМЕНТЬЕВ К.Е. Компьютерные вирусы и антивирусы: взгляд программиста Москва: ДМК-Пресс, 2013. 656с. стр 27
или воздушно-капельных инфекций среди высших  животных.   %REF: [1] https://cyberleninka.ru/article/v/sluchaynye-grafy-kak-model-sredy-rasprostraneniya-i-vzaimodeystviya-samorazmnozhayuschihsya-obektov
Согласно К.Е. Климентьеву, %REF: https://cyberleninka.ru/article/v/sluchaynye-grafy-kak-model-sredy-rasprostraneniya-i-vzaimodeystviya-samorazmnozhayuschihsya-obektov
для подобных эпидемий характерны следующие черты:
\begin{itemize}
	\item Постоянные изменения топологии среды моделирования в связи с высокой мобильностью  объектов;
	\item ограниченный радиус <<заражения>>, как для биологических агентов, так и для компьютерных вирусов, обусловленный физическими свойствами оных.
\end{itemize}


Для моделирования среды существования таких объектов используется класс сетей, называемый <<специальным>> (<<Ad hoc>>), представляющий собой множество случайных графов с разнообразной топологией. При этом нетрудно выделить общие для таких графов характеристики: 
\begin{itemize}
	\item Вероятностное распределение степеней вершин, обозначаемое $k_i$;
	\item также вероятностное распределение для локальных коэффициентов кластеризации вершин, обозначаемое $c_i$.
\end{itemize}
При этом под <<степенью вершины>> 	$x$ подразумевается количество вершин графа $G$, инцидентных вершине $x$, т.е степень вершины указывает количество вершин, непосредственно соединенных только один ребром с данным. %REF: Дистель, Рейнхард (2005), Graph Theory (3rd ed.), Berlin, New York: Springer-Verlag, ISBN 978-3-540-26183-4. , стр 5
<<Локальный 	коэффициент кластеризации вершины>> понимается в том же смысле, что и у К.Е. Климентьева и Б. Хогана %REF : https://cyberleninka.ru/article/v/sluchaynye-grafy-kak-model-sredy-rasprostraneniya-i-vzaimodeystviya-samorazmnozhayuschihsya-obektov
%REF: Hogan, B., Carrasco, J., & Wellman, B. (2007). Visualizing personal networks: Working with participant aided sociograms. Field Methods, 19(2), 116-144.
, т.е <<коэффициент кластеризации>> рассматривается как вероятность того, что два ближайших соседа этого узла сами являются ближайшими соседям, а ЛККВ является мерой того, насколько хорошо связанны связаны между собой соседи данного узла и рассчитывается как отношение
числа связей межу соседями данного узла к возможному числу связей между соседями  (по Хогану) или, менее формально, по К.Е Климентьеву: <<доля <<треугольников>> данной вершины, образованных из <<соседей>> данной вершины и являющихся <<соседями>> друг для друга в общем количестве потенциально возможных треугольников, где $c_{max}=k_i\dfrac{k_i-1}{2}$ >>.


В качестве модели пространства, используемых для имитации развития и распространения эпидемии, применяются неориентированные, маркированные графы (в смысле частного случая сети Петри, в которой каждая позиция входом и выходом точно для одного перехода %REF: http://publ.lib.ru/ARCHIVES/P/PITERSON_Djeyms/_Piterson_Dj..html, стр 200) 
), являющиеся частным случаем дистанционных графов, при этом дистанционный граф понимается как $G = (V,E)$ -- n-мерный дистанционный граф  (граф расстояний), если $V \subseteq \mathbb{R}^n, E \subseteq \{\{\overline{x}, \overline{y} \}: \overline{x}, \overline{y} \in V, |\overline{x} - \overline{y}| = a, a \in \mathbb{R} > 0 \}$, то есть множество вершин $M$ является подмножеством или совпадает с  $n$-мерным пространством,а множество ребер $E$ является подмножеством или совпадает с множеством всевозможных пар вершин $\overline{x}, \overline{y}$, таких, что  евклидово расстояние между $\overline{x}, \overline{y}$ равно некоторому фиксированному вещественному положительному заранее заданному $a$. 

Топология таких графов связана с принципом их построения, т.е каждая вершина графа имеет некоторые пространственные координаты (обычно в $\mathbb{R}^2$ или $\mathbb{R}^3$) и <<соседними вершинами>> считаются только те, расстояние до которых  меньше заданного или равно заданного $r_0$ (расстояние передачи вирусного воздействия). 

Допущения:
\begin{itemize}
	\item Инфекция передается только между инцидентными вершинами за некоторое время;
	\item вершины графа активно перемещаются, постоянно изменяя его конфигурацию.
\end{itemize}
Входные параметры модели:
\begin{itemize}
	\item Начальная конфигурация графа, т.е количество и координаты вершин вместе с их начальным состоянием, обычно в виде перечисления 	<<здоров>>, <<вакцинирован>>, <<болен>>, однако допускается и большее число состояний, подобное применяемому в агентных подходах;
	\item максимальное расстояние инфицирующего воздействия $r_o$;
	\item правила или порядок перемещения вершин в процессе моделирования.
\end{itemize}

Способы построения исходного графа, исходя из <<геометрических>>  и <<географических>> по К.Е Климентьеву соображений подробно рассмотрено в %REF: http://climentieff.ssau.ru/download/Climentieff_Ufa_2012.pdf 
и %REF: https://cyberleninka.ru/article/v/sluchaynye-grafy-kak-model-sredy-rasprostraneniya-i-vzaimodeystviya-samorazmnozhayuschihsya-obektov
. При этом для построения геометрического графа по заранее заданным параметрам типа $k_i$ и $c_i$ используется случайный граф Радо (RRG) или случайный геометрический граф (RGG) и в этой же работе %REF:https://cyberleninka.ru/article/v/sluchaynye-grafy-kak-model-sredy-rasprostraneniya-i-vzaimodeystviya-samorazmnozhayuschihsya-obektov
выводится функция распределения  между двумя случайными точками в единичном квадрате и математическое ожидание расстояния.

Намного более интересным свойством данных моделей являются правила перемещения вершин графа: если в описанных ранее подходах перемещения вершин присутствуют скорее в силу необходимости для динамичности во времени и модель учитывает только их положение по отношению к другим вершинам и, иногда, скорость движения этих, а топология графа не является предметом исследования, то в данном подходе топологию графа и правила перемещения являются ключевым  элементом  при построении модели. В результата построения множества моделей, использующих данный подход, можно сделать следующие выводы, описывающие правила перемещения вершин в графе:
\begin{itemize}
	\item Классические модели блуждания, описывающие, например,  броуновское движение, мало пригодны для описания передвижения людей, особенно движущихся вне рутинного маршрута, а высокоподвижных агентов, в силу своей слабой стационарности. Следовательно, при анализе  движений по плоскости, направление движения распределено равномерно на интервале $[0; 2\pi]$, что показано в %REF: Rhee I. et al. On the Levi Walk Nature of Human Mobility: Do Humans Walk Like Monkeys // IEEE/ACM Transaction on Networking, Vol. 20.	- pp. 630-643.
	;
	\item скорость и продолжительность движения распределены по Леви. При этом <<перемещение>> или <<движение>> понимается как  самый длинный прямолинейный  переход объекта с одного места в другое без паузы или изменения направления %REF: http://repo.ssau.ru/bitstream/Informacionnye-tehnologii-i-nanotehnologii/Modelirovanie-peredvizhenii-uzlov-DTN-seti-s-ispolzovaniem-principa-naimenshego-deistviya-pri-vybore-lokacii-posesheniya-62586/1/itnt_2015_61.pdf
	;	
	\item тем не менее, перечисленные правила перемещения актуальны только для перемещениям на ограниченном участке плоской(или приближенной к плоской)  поверхности без учета препятствий, конфигураций помещения, уставившихся маршрутов, столкновений с другими телами и необходимости их избегать. Для внесения корректировок, учитывающих такие особенности, необходим сбор статистики на множестве реальных наблюдений с последующим внесением в некоторую  ГИС. 
\end{itemize} 

Базируясь на работах %REF: Rhee I. et al. On the Levi Walk Nature of Human Mobility: Do Humans Walk Like Monkeys // IEEE/ACM Transaction on Networking, Vol. 20.	- pp. 630-643.
,
%REF: http://repo.ssau.ru/bitstream/Informacionnye-tehnologii-i-nanotehnologii/Modelirovanie-peredvizhenii-uzlov-DTN-seti-s-ispolzovaniem-principa-naimenshego-deistviya-pri-vybore-lokacii-posesheniya-62586/1/itnt_2015_61.pdf
,
%REF: http://climentieff.ssau.ru/download/Climentieff_Ufa_2012.pdf
,
%REF: http://www.ssc.smr.ru/media/journals/izvestia/2016/2016_4_744_748.pdf стр 744
можно сделать следующие выводы, касающиеся основных свойств данного подхода к моделированию: 
\begin{enumerate}
	\item Данный подход хорошо подходит для моделирования распространения инфекционных заболеваний среди групп людей и животных, особенно блуждающих, в т.ч. по Леви, в некотором постоянном данном ареале;
	\item использование данных моделей улучшает понимание того, какие графы лучше описывают топологию быстро движущихся источников инфекции. Так, в работах %REF: http://climentieff.ssau.ru/download/Climentieff_Ufa_2012.pdf 
	и % REF: https://cyberleninka.ru/article/v/sluchaynye-grafy-kak-model-sredy-rasprostraneniya-i-vzaimodeystviya-samorazmnozhayuschihsya-obektov 
	четко прослеживается зависимость между <<Ad hoc>> графами и графами Радо и той картиной, которую дают реальные перемещения мобильных устройство при вспышке компьютерных инфекций.
\end{enumerate}


Достоинства данного подхода: высокая скорость работы моделей при правильном выборе инструментария(готовые фреймворки для работы с графами); высокая точность предсказания протекания вспышки инфекции на начальном этап этой вспышки; наличие уже разработанных алгоритмов, приборов и готовых собранных данных для занесения в многослойные ГИС; возможность модификации алгоритмов для учета рельефа местности и возможных столкновений инфицируемых  сущностный; математический аппарат блужданий Леви допускает модификации, позволяющие легко настраивать и изменять принципы перемещения сущностней; сопоставление результатов моделирования с реальными процессами позволяет изучать зависимости между некоторыми специальными графами, описанными выше, и путями распространения инфекций. 


Недостатки: в силу специфики требований, предъявляемых к моделям данного типа, точное моделирование возможно только в течении небольшого количества итераций модели, т.к предполагается характер <<вспышки>>, то есть быстрого бурного увеличения числа зараженных индивидуумов с достаточно быстрым падением числа больных при изобретении лекарства или разработке антивируса; для построения точных моделей требуется не только точно знать координаты физического пространства, на котором будет происходить распространения заболевания, но и иметь детальные модели самих поверхностей или рельефов, что необходимо для точного построения блужданий Леви; используемые графы высокодинамичные и могут требовать значительных затрат как по памяти, так и по процессорному времени.

\subsection{Модели инфекционной динамики на основе предфрактальных графов}

Для моделирования распространения инфекционных заболеваний можно также использовать предфрактальные графы.В основу данного подхода к моделированию положена идея о том, что графы, использующие операцию "замены вершины затравкой" (ЗВЗ), которая будет рассмотрена далее, <<прирастают>> не отдельными вершинами, а самоподобным или приближенно самоподобным  графом, т.е фрактальным, хорошо описывают   системно <<прирастающие>> структуры: телекоммуникационные, инфраструктурные, социальные и технические сети, некоторые популяции, в которых существует разбиение на группы, в которых связей внутри группы больше, чем между отдельными группами. При этом показано,  что такие модели могут быть использованы для моделирования распространения инфекционных заболеваний, а именно, что процессы разрастания  графов релевантны динамике распространения инфекционных  заболеваний  и распад таких графов соответственно релевантен  спаду эпидемий %REF: https://new-disser.ru/_avtoreferats/01005400379.pdf 
Математико-алгоритмическая  идея построения фрактальных графов -- операция замены вершины затравкой (ЗВЗ). Для этого вводится несколько понятий, рассмотренных ниже.
По определению, динамический граф ${G_D}$ -- последовательность обычных графов $G_l$, не имеющих параллельных ребер и петель. Затравка -- какой-либо произвольный связный граф, т.е граф, в котором между любой парой вершин существует хотя бы один путь, сама затравка обозначается как  ${H=(W,Q)}$. Предфрактальный граф обозначается как $G_L=(V_L,E_L)$, определяется рекуррентно, заменяя каждый раз в построенном на предыдущем этапе $l = \overline{1, L-1}$ графе $G_l=(V_l,E_l)$ каждую вершину затравкой $H$. Вершины могут соединяться случайно или по заданному правилу или в некоторой заранее заданной последовательности. На этапе $l=1$ предфрактальному графу соответствует затравка $G_1=H$. Говорят, что предфрактальный граф $G_L$  \textit{порожден} затравкой $H$, при этом процесс построения предфрактального графа $G_L$  по сути является процессом построения последовательности других предфрактальных графов $\overline{G_1,G_L}$, называемый <<траекторией>>. 

Суть операции замены вершины затравкой (ЗВЗ): в графе $G=(V, E)$ из выбранной для замещения затравкой вершины $\tilde{v}  \in V $ выделяется множество $\tilde{V} = \{\tilde{v}_j\} \subseteq V$, где j = $\overline{1,|\tilde{V}|}$, смежных ей, вершине $\tilde{v}  \in V $, вершин; затем из графа $G$ удаляется вершина $\tilde{v}$ и все инцидентные ребра, т.е ребра, для которых эта вершина общая; после $\forall$ $\tilde{v}_j \in \tilde{V}, j=\overline{1,|\tilde{V}|}$ соединяется ребром с одной из вершин затравки $H$, при этом соединение может происходить случайно или по заранее заданному правилу или в заданной последовательности.

Под <<распознаванием предфрактального графа>> в таких подходах к моделированию  обычно понимается  определение траектории построения предфрактального графа при условии, что заданы виды и типы затравок. При этом задача распознавания различных типов предфрактальных графов хорошо изучена и отработана в практических алгоритмах, что дает широкую теоретико-практическую базу для решения данной задачи. %REF: https://cyberleninka.ru/article/v/raspoznavanie-predfraktalnyh-grafov-s-zatravkoy-udovletvoryayuschey-usloviyu-ore
%REF: Найманова И.Х., Кочкаров А.М. Об одной задаче распознавания предфрактального графа // Вестник Самарского государственного технического университета. - 2007. - № 1 . - С. 194-196; 
%REF: Утакаева И.Х. Алгоритм распознавания предфрактального графа с затравкой регулярной степени // Обозрение прикладной и промышленной математики. - 2008. -Том 15.- Выпуск3. - С. 531-533; 
%REF: Утакаева И.Х., Кочкаров А.М. Моделирование процесса распространения эпидемии и нахождения возможных очагов заражения на предфрактальном графе // Сборник трудов 111-ей Всероссийской научно-практической конференции «Перспективные системы и задачи управления». - Таганрог: Издательство Таганрогского технологического института ЮФУ, 2011. - С.273-283 

В предфрактальном графе $G$ рёбра, порожденные на этапе $l$, где $l \in {1,2,3, ..., L}$ называются ребрами ранга $l$, при этом <<новыми>> ребрами называют ребра ранга $L$ в предфрактальном графе $G_L$, все остальные называются <<старыми>>.  При этом, под рангом вершины графа понимается наименьший ранг $l$, выбираемый среди всех рёбер, инцидентных для данной вершины. Вершина ранга $l$ обозначается как $v^{(l)}$, $l \in \{1,2,..., L\}$. Также вводится понятие очага заражения ранга $l$, который обозначается как $\tilde{v}^{(l)}$ и является, по сути, вершиной ранга $l$. 

Одной из самых важных задач данного подхода к моделированию является задача распознавания фрактального графа, при этом различают два вида распознавания: неявное -- определение фракталньости графа и установление некой $n$-вершинной затравки, на которой он базируется; явное -- представление множества рёбер для всякого ранга в явном виде или же представление порождающей траектории в явном виде для некоторого заданного графа $G$. Для этого используется несколько алгоритмов, показанных в %REF: https://new-disser.ru/_avtoreferats/01005400379.pdf
и 
%REF: https://cyberleninka.ru/article/v/raspoznavanie-predfraktalnyh-grafov-s-zatravkoy-udovletvoryayuschey-usloviyu-ore
и обозначаемых $\alpha_1 ... \alpha_4$ %REF: https://www.dissercat.com/content/issledovanie-svoistv-i-raspoznavanie-predfraktalnykh-grafov/read
.

Как показано в %REF: https://cyberleninka.ru/article/v/otsenka-diametra-oblasti-rasprostraneniya-virusov-po-modelyam-na-predfraktalnyh-grafah 
и 
%REF: https://new-disser.ru/_avtoreferats/01005400379.pdf
, данные модели могут быть использованы для  описания распространения  и протекания как  инфекционных заболеваний, так и для эпидемий <<компьютерных вирусов>>, особенно в локальных сетях, т.к их топография зачастую подобна предфрактальным или фрактальным графам; модели, которые можно построить с помощью данного подхода, могут быть использованы для выявления кластеров заражения или очагов заражения (для эпидемий, не предполагающих иммунитета). Также, значительным преимуществом данных моделей является то, что для распознавания фрактального графа  имеется набор готовых алгоритмов $\alpha_{1...4}$, являющихся полиномиальными; модели позволяют учитывать уровень иммунитета каждого человека или каждого устройства в сети; уже существуют готовые реализации данных моделей в виде программных реализаций. Также, для прикладной реализации важной является   вычислительная сложность, для $\alpha_2 = O(|E|L)$ (если <<старые>> ребра не пересекаются) и  $\alpha_{1,3,4} \leqslant O(|E|L)$, где $|E|$ понимается как количество ребер. 

Недостатком данного подхода можно назвать то, что в силу конечного числа шагов трудно добиться полной фракталньости графа и зачастую используются приближенно фрактальные модели.


\subsection{SIR - SЕIFDR модель}

SIR модель впервые была предложена Кермаком и МакКендриком в 1927 г. для описания заразных заболеваний в закрытых популяциях с течением времени. Модель предполагает фиксированный  размер популяции (без смертей от болезни и естественных причин, без рождаемости), мгновенность инкубационного периода возбудителя болезни, равную продолжительность заразности и самой болезни, полною однородность популяции в терминах возраста, пола, пространства или социальной структуры, т.е данные параметры не учитываются. Модель состоит из системы нелинейных дифференциальных уравнений: \\
\begin{equation} \label{SIR_model:1}
	\begin{cases}
		\dfrac{dS}{dt} =  -\beta SI \\
		\dfrac{dI}{dt} = \beta SI - \gamma I \\
		\dfrac{dR}{dt} = \gamma I \\
	\end{cases}			
\end{equation}
, где $t$ -- время, $S(t)$ -- количество подозреваемых(Suspected), $I(t)$ -- инфицированных(Infected), $R(t)$ -- выздоровевшие с иммунитетом (Recovered), $\beta$ -- интенсивность инфицирования, $ \gamma $ -- интенсивность выздоровления. Важным параметром, управляющим развитием модели, является эпидемиологический порог $R_0 = \dfrac{\beta S}{\gamma} $. Если $R_0 < 1$, то один больной человек заразит  менее одного человека перед своим излечением, т.е вспышка иссякнет; если $R_0 > 1$, то инфекция будет распространяться, т.е $\dfrac{dI}{dt} > 0$. %REF: http://mathworld.wolfram.com/Kermack-McKendrickModel.html
% и прочие ссылки с нижней части этой страницы.
Данная модель и ее частная версия Кермака-МакКендрика настолько интересны, насколько и просты, что делает их подходящими только для описания  коротких вспышек не летальных заболеваний, например, холеры в Лондоне в 1865 г. Для описания других болезней необходимо исследовать их особенности и, в первую очередь, состояния, которые свойственны больным. Так возможно  модифицировать базовую SIR модель до модели, учитывающей смертности и заразность трупов до их захоронения. Такая модель называется SEIFDR модель и подробна исследована  в %REF: Быкова. Мультиагентный подход
и является глубокой модификацией SIR, где вводятся три новых состояний и ряд переменных, от которых динамика этих состояний зависит. Так, SEIFDR модель описывается системой:
\begin{equation} \label{SEIFDR_model:1}
	\begin{cases}
		\dfrac{dS}{dt} = -\beta_I S I + \beta_F S F \\
		\dfrac{dE}{dt} = \beta_I S I + \beta_F S F - \alpha E \\
		\dfrac{dI}{dt} = \alpha E - I \gamma_D ( (1 - \delta_D) + \gamma_D \delta_D) \\
		\dfrac{dF}{dt} = \gamma_D \delta_D I - \gamma_F F \\
		\dfrac{dR}{dt} = \gamma_I(1-\delta_D) I + \gamma_F F
	\end{cases}
\end{equation}
,  где $\beta_I$ и $\beta_F$ -- интенсивность контактов между людьми и на похоронах, $\alpha$ --  интенсивность инкубационного периода, $\gamma_I$ интенсивность выздоровления, $\gamma_D$  интенсивность умирания агентов, $\gamma_F$ --   интенсивность похорон, $\delta_D$ -- вероятность смертности. 

Преимущества: чистый математический подход, выраженный СДУ; наличие большого количества готового имитационного ПО, способного быстро и достаточно точно решать \eqref{SEIFDR_model:1}; высокая прогностическая точность при правильном подборе параметров для заданной болезни; возможность динамически расширять возможные состояния в будущем.

Недостатки: система требует длительной калибровки параметров для максимально точного описания болезни; требуется минимально зашумленная статистика хотя бы инфицированным и умершим для сравнения с работой модели; после калибровки модели для данной болезни на одном городе необходимо тестировать откалиброванную модель на другом городе для избежания создания модели, которая будет только <<подогнанной>> под конкретный случай, но не имеющей прогностических свойств; каждая новая болезнь требует новой калибровки с нуля; модель не учитывает социальную структуру популяции.
\subsection{Выводы о существующих моделях}


Как было показано выше, существует широкое множество различных подходов к имитированию распространения инфекционных заболеваний. При этом каждый из таких подходов обладает своим собственным рядом преимуществ и недостатков. Для реализация и сравнения выбраны две модели: мультиагентная модель, за простоту и скорость, и SEIDFR модель за её строгость и гибкость.
	\newpage
\parindent=1cm %красная строка
\begin{center}
		
		\section{Реализация компьютерных моделей распространения инфекционных заболеваний}
		
\end{center}

Как уже было указано  в первой главе, для реализации выбраны мультиагентная и SEIFDR модели.

После выбора средств моделирования, необходимо, согласно целям работы, выбрать целевое заболевание и среду моделирования.


\subsection{Выбор заболевания, его краткое описание}
Прежде всего, исследуемое заболевание должно обладать ярко выраженными симптомами, что позволяет получить высокорепрезентативную  выборку.


В качества целевого заболевания для анализа выбрана геморрагическая лихорадка Эб$\acute{o}$ла -- острое вирусное высококонтагиозное заболевание, поражающие человека и других приматов, . Данное заболевание характеризуется высокой заразностью (человеческие останки остаются заразными до 50 дней 
%REF: https://www.cdc.gov/media/releases/2014/t0728-ebola.html
\cite{CDC:2014:1}), коротким инкубационным периодом, резкими клиническими проявлениями и очень высокой смертностью(до 90\%, в среднем около 50\% ) \cite{WHO:2018:1}, \cite[с. 444]{Sunit:2014}. %REF: https://www.who.int/en/news-room/fact-sheets/detail/ebola-virus-disease
%REF: https://books.google.com/books?id=l5MtJdDhie0C&pg=PA444#v=onepage&q&f=false Singh, Sunit K.; Ruzek, Daniel, eds. (2014). Viral hemorrhagic fevers. Boca Raton: CRC Press, Taylor & Francis Group. СТР 444
Инфицирование происходит при прямом контакте через биологические жидкости, выделения, предметы. Основные симптомы (по времени проявления):  усталость, лихорадка, слабость, миалгия,  артралгия, боли в горле, чиханье и диарея,  обширные внутренние и внешние кровотечения, в том числе коагулопатия, кровохарканье, кровь в глазном белке, гиповолемический шок, приводящие к смерти \cite{ncbi:2011}, \cite{ovid:2016}, \cite{Springer:2015}  . 
%REF: https://www.ncbi.nlm.nih.gov/pmc/articles/PMC3406178/
%REF: https://insights.ovid.com/crossref?an=00055735-201611000-00011
%REF: https://link.springer.com/article/10.1007%2Fs10620-015-3691-z

Данное заболевание удобно для изучения тем, что распространяется в странах с низким уровнем гигиены и медицинской поддержки, что позволяет изучить <<чистый>> вирус на <<чистой>> популяции,  т.е. модель, описывающая вирус, может пренебрегать медицинским мероприятиями; также имеется подробная посуточная статистика с низким уровнем шума по государствам и отдельным городам.


Типичные группы риска: медработники, участники похоронных команд или другие лица с непосредственным доступом к трупу, члены семьи и другие лица, находящиеся в тесном контакте с инфицированным, ведущие промысел во влажных экваториальных лесах охотники, при любом контакте с трупами инфицированных животных или при использовании зараженных животных в пищевой промышленности. Также в этой группе находятся все жители Центральной и  Южной (кроме ЮАР) Африки  и Западного и Восточного  побережий -- в этих регионах наблюдается низкий уровень санитарии, небезопасные промыслы на диких животных, низкий уровень медицины и образования людей в целом и неблагоприятная социально-политическая ситуация, что особенно хорошо видно на примере вспышки лихорадки Эбола в ДРК весной 2019 г. \cite{WHOReport:25apr2019}, \cite{WHOReport:02may2019}, \cite{WHOReport:09may2019}.
%REF: https://www.who.int/csr/don/25-april-2019-ebola-drc/en/
%REF: https://www.who.int/csr/don/02-may-2019-ebola-drc/en/
%REF: https://www.who.int/csr/don/09-may-2019-ebola-drc/en/


\subsection{Сбор  и анализ статистики по местности}

Прежде всего, следует отметить, что низкий уровень инфраструктуры и подготовки медицинского персонала на местах вместе с большими расстояниями значительно затрудняет и <<зашумляет>> сбор статистики, поэтому среди источников для анализа необходимо выбрать наиболее авторитетный и полный, что снимет необходимость интерполяции данных и фильтрации выборки от шумов.


Усилия WHO направлены на детальный ежесуточный сбор статистики по государством и отдельным городам. Основываясь на анализе репозитория WHO \cite{github_ebola_data:2014} %REF: https://github.com/cmrivers/ebola/
, можно утверждать, что самая детальная статистика дана по государству Сьерра-Леоне и наиболее пострадавшие города, также имеющие самую репрезентативную статистику, это Кайлахун и Кенема, что соответствует выводам \cite{Bykova:2015}%REF: Быкова. Мультиагентный подход 
.

\subsection{Общие допущения для моделей}
Как показано выше, лихорадка Эбола является высококонтагиозным заболеванием с высокой летальностью. Рассмотрим основные этапы данного заболевания в терминах симптомов и продолжительности, что важно при реализации моделей:
\begin{enumerate}
	\item Заражение. Происходит при контактах с телесными жидкостями зараженного человека или примата.
	\item Инкубационный период. Занимает от 4 до 10 дней \cite{WHO:2018:1}, \cite{CDC:Signs}, \cite{TheJournalOfMedicine:2014}, \\ \cite{ncbi:2014}  . %REF: https://www.who.int/en/news-room/fact-sheets/detail/ebola-virus-disease
	%REF: https://www.cdc.gov/vhf/ebola/symptoms/index.html
	%REF: https://web.archive.org/web/20141129144852/http://www.njmonline.nl/getpdf.php?t=a&id=10001148
	%REF: https://www.ncbi.nlm.nih.gov/pmc/articles/PMC4205154/
	\item Явное проявление болезни. Сопровождается усталостью, жаром и болями в теле. На данном этапе наиболее важными  и опасными симптомами являются гематемезис, гемоптизис, кровотечения в ЖКТ и мелена, петехии. Данные симптомы представляют опасность здоровью инфицированного ввиду гиповолемическиго шока, но и являются основной причинной передачи заболевания между людьми \cite{NYTM:2014}, \cite{CDC:Transmission}. %REF: https://well.blogs.nytimes.com/2014/10/03/ebola-ask-well-spread-public-transit/
	%REF: https://www.cdc.gov/vhf/ebola/transmission/index.html
	\item Выздоровление. Обычно наступает через 1-2 недели после проявления симптомов 
	\cite[с. 332]{Hunter:2013}. %REF: https://books.google.com/books?id=UgvdM8WRld4C&pg=PA332#v=onepage&q&f=false
	Считается, что после этого вырабатывается иммунитет сроком до 10 лет, что позволяет допускать при построении модели, что лица, перенёсшие заболевание, вырабатывают стойкий иммунитет. Однако неизвестно взаимодействие данного специфического иммунитет с другими штаммами лихорадки Эбола \cite{CDC:Transmission}. %REF: https://www.cdc.gov/vhf/ebola/transmission/index.html?CDC_AA_refVal=https%3A%2F%2Fwww.cdc.gov%2Fvhf%2Febola%2Ftransmission%2Fqas.html
	На данный момент ведутся исследования в области сохранения вирулентных клеток у пострадавших даже после выздоровления. В силу спорности данного утверждения и отсутствия значимой статистки, описываемые далее модели не будет учитывать этот источник заражения \cite{WHO:semen:2016}, \cite{ScienceTime:2015}. %REF: https://www.who.int/reproductivehealth/topics/rtis/ebola-virus-semen/en/
	%REF: http://www.sciencetimes.com/articles/6000/20150502/ebola-transmitted-through-sex.htm
	\item Смерть. Обычно наступает через 6-16 дней после проявления первых симптомов из-за гиповолемическиго шока \cite{MicroBio:2014}. %REF: https://jgv.microbiologyresearch.org/content/journal/jgv/10.1099/vir.0.067199-0
	Важно заметить, что умершие тела все еще являются источником заражения, поэтому вероятность контакта между умершими и здоровыми людьми учитывается в описываемых далее моделях.
\end{enumerate}
Важно заметить, что в нулевой момент времени вся популяция не  может находиться в состоянии <<Suspected>>, иначе распространение болезни происходить не будет,  т.е. в модели вводится параметр, отвечающий за начальное количество людей, находящимся в инкубационном периоде,  т.е. в состоянии <<Exposed>>  в терминах SEIFDR модели.   

Состояния, в которых находятся агенты, можно описать с помощью следующего стейтчарта:
\begin{figure}
	\centering{\includegraphics[scale=0.5]{Images/SEIFDR_chart.png}}
	\caption{Стейтчарт, описывающий состояния агентов в SEIFDR и мультиагентной моделях.}
\end{figure}
\subsection{Выбор средств моделирования}

Для построения SEIFDR модели требуется имитационное ПО, способное решать системы дифференциальных уравнений и гибкого визуализировать результаты моделирования, а также сопоставлять их с реальными результатами. При этом ПО не должно быть привязано к конкретной популяции, но должна быть возможность точно и гибко калибровать модель. Для анализа были взяты NetLogo, SOARS  и AnyLogic. NetLogo был отброшен,   т.к. является ПО для обучения и  требует модификации для сложных вычислений и экспериментов на внешних языках программирования. SOARS также был отброшен,   т.к. не имеет возможности решать системы дифференциальных уравнений. Оптимальным выбором был признан AnyLogic,  т.к. данное ПО является популярным промышленным решением, обладает методами для численного решения систем дифференциальных уравнений, инструментами для проведения оптимизационных экспериментов и инструментами для построения графических интерфейсов пользователя.


Для построения мультиагентной модели был выбран язык программирования Python в связке с библиотеками NetworkX %REF: https://networkx.github.io/
\cite{NetworkX:git} , используемой для построения графа локаций 
и WolframClient 1.0.2 %REF:https://pypi.org/project/wolframclient/
\cite{PiP:Wolfram}, используемой для прямого взаимодействия с системой компьютерной алгебры Wolfram Alpha, что позволяет отказаться от изучения и использования NumPy и SciPy %REF: https://www.wolframalpha.com/about/
.

\subsection{Построение SEIFDR модели и ее оптимизация}
Данная модель представляет собой систему дифференциальных  уравнений %REF: http://mathworld.wolfram.com/Kermack-McKendrickModel.html
\cite{Wolfram_MW:SIR} и имеет ряд переменных, описывающих вероятность контакта и смерти, является детерминированной. Очевидно, что каждое заболевание имеет свой характер и, следовательно, свое значение этих параметров. Таким образом, основной задачей при реализации данной модели является проведение ряда экспериментов с целью подбора этих параметров с последующим тестированием на другой популяции, что позволит подтвердить правильность подбора данных параметров. Результатом работы модели будет  количество больных, умерших и выздоровевших в каждый дискретный момент времени.

При построении данной модели возможно использовать средства AnyLogic из палитры <<Системная динамика>>: количество людей в каждом стоянии описывается одноименным накопителем, а переменные, отвечающие за интенсивность контактов, вероятность смерти и т.д., представляют собой статические параметры. В силу простоты модели и незначительного количества связей между параметрами и накопителями, вместо использования потоков возможно применить связи -- это не оказывает влияния на решение системы дифференциальных уравнений. 


%IMG: скрин модели 
\begin{figure}[H]
	\centering{\includegraphics[scale=0.5]{Images/AnyLogicModel}}
	\caption{Общий вид SEIFDR модели в среде AnyLogic.}
\end{figure}


После построения базовой модели с неоткалиброванными параметрами необходимо ввести целевую функцию, минимизация которой и является ключевым этапом калибровки модели. Согласно  %REF: Быкова,  стр 26
\cite[с. 26]{Bykova:2015}, в качестве  целевой функции может быть выбрано среднеквадратичное отклонение результатов работы модели от  реально наблюдаемых результатов для больных и  умерших  по отдельности. Введение целевой функции позволяет численно,  т.е. аналитически, сравнивать результаты запусков модели с различными параметрами.
После введения целевой функции появляется возможность провести стохастический и оптимизационный эксперименты.

Стохастический эксперимент является базовым на начальном этапе настройки модели,  т.к. многократным повторением эксперимента достигается возможность получения достоверных выводов о работе модели в целом. Очевидно, что при наличии стохастических  параметров в  модели, недостаточно будет одного прогона. Количество итераций обосновано в  %REF: КондратьевM.A.Разработкамоделираспространенияинфекционныхзаболеванийнаосновеагентногоподхода.––Санкт-Петербург, 2012. 
\cite{Kondratiev:2010} и было использовано при проведении данного эксперимента. 

Следует отметить, что отсутствие стохастических параметров делает стохастический эксперимент бесполезным для данной модели и данный вид эксперимента будет проводиться только в мультиагентной модели.

Идея  оптимизационного  эксперимента заключается в многократном выполнении модели с итеративным изменением параметров, подлежащих уточнению, для улучшения модели. Для эксперимента использовались средства AnyLogic  и системы компьютерной алгебры Wolfram Mathematica. По окончанию эксперимента возможно наблюдать лучшую итерацию, её оптимизированные параметры, значение целевой функции для каждого интересующего нас состояния.

\subsection{Построение мультиагентной модели и её оптимизация}

Данная модель является стохастической и  представляет собой динамично взаимодействующее множество агентов двух типов: агентов-людей и агентов-локаций. 

Множество агентов-локаций описывает территорию, где происходит действие, при этом необязательно учитывать реальный геодезический или географический характер местности, а достаточно нормально распределить агентов-людей по локациям, что будет соответствовать реальному распределению людей в городах между общественными местами (школы, больницы, предприятия) и местами жительства.    Агенты-люди описывают людей с их восприимчивостью, возрастом и текущей позицией и являются основным действующим лицом в модели,  т.к. именно они перемещаются по локациям и участвуют в событии контакта, которое может привести к заражению. Состояния агентов-людей полностью соответствуют таковым у SEIFDR модели, однако, механизм перехода по состояниям значительно отличается и не описывается дифференциальными уравнениями. Как было указано в разделе 2.3, вся популяция изначально полагается в стоянии  <<Suspected>> и, подобно SEIFDR модели, имеется некоторое случайное число агентов с заболеванием в инкубационном периоде, распределенное аналогичным образом. Информация, инкапсулируемая классом <<Агент-человек>> используется в вычислениях: так, возраст влияет на вероятность попадания в новую локацию (старики и дети менее подвижны) и восприимчивость к болезни (иммунитет детей и стариков развит меньше %REF: https://applied-research.ru/pdf/2011/06/2011_06_047.pdf
\cite{Parahonsky:2011}).

Принцип взаимодействия элементов модели и некоторые числовые параметры выбраны на основании уже проведенных исследований, связанных с лихорадкой Эбола
%REF: http://currents.plos.org/outbreaks/article/obk-14-0043-modeling-the-impact-of-interventions-on-an-epidemic-of-ebola-in-sierra-leone-and-liberia/
\cite{Plos_Outbreak:1}, \cite{Plos_Outbreak:2}. 

Агенты-люди распределены по локациям и каждый день раз в 8 часов перемещаются между ними случайным образом, что примерно соответствует суточной активности  человека. При этом между агентами, находящимися в одной локации, возможно событие контакта  и передачи инфекции, что будет описано подробно далее. 

Основной событийный цикл модели выглядит следующим образом:
\begin{enumerate}
	\item В каждой локации для каждого агента в состояниях <<Exposed>>, <<Infected>> или <<Funeral>>  с некоторой вероятностью происходит контакт с другим   здоровым агентом,  т.е. <<Suspected>> агентом. Вычисляется вероятность заражения жертвы. Если событие заражения произошло, жертва переходит в состояние <<Exposed>>.
	\item Раз в день проверяется событие перехода по зараженным состояниями. Переход между этими состояниями происходит с заданной для каждого состояния интенсивностью. Так, переход из <<Exposed>> в <<Infected>> происходит в среднем раз в 7 дней %REF: https://www.ncbi.nlm.nih.gov/pmc/articles/PMC4205154
	\cite{ncbi:2014} ,  <<Infected>> -- <<Funeral>> -- 10.5 дней, <<Infected>> -- <<Recoverd>> -- 16 дней. Переход <<Funeral>> -- <<Dead>> от интенсивности похорон,  т.е. является специфичным для каждого государства и даже города в зависимости от мобилизованных ресурсов и общего уровня здравоохранения, но  т.к. данная модель обучалась на достаточно бедном и неразвитом государстве в период, предшествующий мобилизации ресурсов, полагается, что похороны могут занимать до 4.5 дней. 
	\item Происходит перемещение агентов по локациям. В общем случае, нельзя утверждать, что локация, в которую перейдёт агент не будет совпадать с той, в которой он находился до этого. Исполнение цикла возвращается в 1. 
\end{enumerate}

Основой для вычислений является эмпирическая формула  вычисления вероятности заболевания:
\begin{equation}\label{stohastic_eq:1}
P_i = 1 - \operatorname{e}^{\tau N_r \ln(1- \beta S_i\rho)} \,,
\end{equation}
где $P_i$ -- вероятность инфицирования  i-го агента, $\tau$  -- коэффициент расстояния, важный при передаче заболевания на большие дистанции, например, при межконтинентальных перевозках, (в данной модели равен 1), $N_r$ -- кол-во зараженных данным заболеванием (в данной локации), $\beta$ -- возрастной коэффициент, за 1 приняты взрослые, $S_i$ -- восприимчивость индивида i (индивидуальный параметр, требующий итеративной калибровки, в данной модели индивиды популяции имеют распределение, близкое к нормальному с мат. ожиданием 0.5 и дисперсией 0.1), $\rho$ -- допустимость,  т.е. вероятность заражения одного индивида другим за фиксированное время ( в данной модели -- одни сутки,  т.е. время пребывания агента в одной локации).

Как и в SEIFDR модели, целевой функцией выбрано среднеквадратичное отклонение реальных результатов от результатов работы модели по  больным и умершим. Как было сказано ранее, в данной модели присутствует несколько стохастических параметров, что делает актуальной для нее стохастический эксперимент.  В целом, гистограмма соответствует аналогичной в \cite[с. 25]{Bykova:2015}

Аналогично предыдущей модели проводится оптимизационный эксперимент. Для выполнения расчетов привлекались средства Wolfram Mathematica.

\subsection{Тестирование и анализ разработанных моделей}

Предполагается, что модели будут использоваться в реальных условиях на современных персональных компьютерах, поэтому обучение и тестирование модели проводились на аналогичной машине: \\
ОС: Windows 10 \\
Разрядность: x64 \\
CPU: Intel(R) Core(TM) i7-8550U CPU @ 1.80GHz (8 CPUs), ~2.0GHz \\
Оперативная память: 16 Гб \\

Для анализа и сравнения результатов работы отдельных итераций каждой модели и обеих моделей между собой, необходимо было не только откалибровать модели по параметрам, но и проверить корректность моделей в целом. Для этого модели сначала обучались на  городе Кенема, а затем тестировались на городе Кайлахун. Сравнение проводилось по графикам заболеваемости и смертности визуально  и аналитически по значению целевой функции для больных и умерших. 

Согласно %REF: Быкова. Мультиагентный подход
\cite{Bykova:2015}, при обучении моделей необходимо учитывать принятие мер по борьбе с распространением заболевания, следовательно, модели будут обучаться на временном интервале до принятия мер.
%IMG: график статистики смертности в Кенеме и Кайлхуне


\begin{figure}[H] 
	\centering{\includegraphics[scale=1.25]{Images/RealDeathsDynamic.png}}
	\caption{Реальная динамика смертности в городах. Кенема и Кайлахун показаны голубым и оранжевым цветом соответственно.}
	\label{RealDeadData}
\end{figure}



Как видно на графике \ref{RealDeadData}, с определенного дня приращение числа больных резко снижается, что соответствует времени принятия мер по защите населения. Исходя из этого, модели будут обучаться на временном интервале, соответствующему  [0; 22] дня на графиках выше.

В таблицах \ref{tab:Param:SEIFDR}, \ref{tab:Marks:InfectedAgent}  приведены параметры для обеих моделей  после окончательной оптимизации.

%TABLE: таблица параметров для SEIFDR 
%TABLE: таблица параметров для мультиагентной 
% Please add the following required packages to your document preamble:
% \usepackage{graphicx}
\begin{table}[H]
	\renewcommand{\thetable}{\thesubsection.\arabic{table}}
	\resizebox{\textwidth}{!}{%
		\begin{tabular}{|l|l|}
			\hline
			Параметр                                                                          & Значение                                       \\ \hline
			Начальная популяция                                                               & 18000                                          \\ \hline
			Suspected(0)                                                                      & 18000-Exposed(0)-Infected(0)                              \\ \hline
			Exposed(0)                                                                        & 25 \\ \hline
			Infected(0)                                                                        & 15 \\ \hline
			\begin{tabular}[c]{@{}l@{}}Интенсивность контактов \\ между больными\end{tabular} & 0.5608                                          \\ \hline
			\begin{tabular}[c]{@{}l@{}}Интенсивность контактов\\ с умершими\end{tabular}      & 0.1481                                          \\ \hline
			\begin{tabular}[c]{@{}l@{}}Интенсивность \\ инкубационного периода\end{tabular}   & 0.073                                            \\ \hline
			Интенсивность выздоровления                                                       & 0.064                                         \\ \hline
			\begin{tabular}[c]{@{}l@{}}Интенсивность \\ умирания\end{tabular}                 & 0.1266                                      \\ \hline
			Интенсивность похорон                                                             & 0.4868                                        \\ \hline
			Вероятность смерти                                                                & 0.873                                           \\ \hline
		\end{tabular}%
	}
	\caption{Откалиброванные параметры SEIFDR модели.}
	\label{tab:Param:SEIFDR}
\end{table}


% Please add the following required packages to your document preamble:
% \usepackage{longtable}
% Note: It may be necessary to compile the document several times to get a multi-page table to line up properly
\renewcommand{\thetable}{\thesubsection.\arabic{table}}
\begin{longtable}{|l|l|}
	
	\hline
	Параметр & Значение \\ \hline
	\endfirsthead
	%
	\endhead
	%
	$\tau$ & 1 \\ \hline
	$S_i$ & \begin{tabular}[c]{@{}l@{}}Нормально распределено \\ на интервале [0.1; 2]\end{tabular} \\ \hline
	\begin{tabular}[c]{@{}l@{}}Начальная \\ популяция\end{tabular} & 18000 \\ \hline
	Suspected(0) & 18000-Exposed(0)-Infected(0) \\ \hline
	Exposed(0) & 25 \\ \hline
	Infected(0)                                                                        & 15 \\ \hline
	$\rho$ & 0.759 \\ \hline
	\begin{tabular}[c]{@{}l@{}}Возрастной\\ коэффициент $\beta$\end{tabular} & \begin{tabular}[c]{@{}l@{}}Возраст         Значение $\beta$\\ 0-14                1.8\\ 15-24             1.73\\ 25-54             1\\ 55-64             1.25\\ 65\textbf{+}                1.6\end{tabular} \\ \hline
	\begin{tabular}[c]{@{}l@{}}Интенсивность \\ выздоровления\end{tabular} & 0.064 \\ \hline
	\begin{tabular}[c]{@{}l@{}}Интенсивность\\ умирания\end{tabular} & 0.1266 \\ \hline
	\begin{tabular}[c]{@{}l@{}}Интенсивность\\ похорон\end{tabular} & 0.4868 \\ \hline
	\begin{tabular}[c]{@{}l@{}}Вероятность\\ смерти\end{tabular} & 0.873 \\ \hline
	
	\caption{Откалиброванные параметры мультиагентной модели}
	\label{tab:Marks:InfectedAgent}
\end{longtable}
На графиках \ref{FinishedModelDead}, \ref{FinishedModelInfected}  представлены результаты работы откалиброванных моделей для города Кайлахун. 

%IMG: график заболеваемости 
%IMG: график смертонсти
\begin{figure}[H]
	\centering{\includegraphics[scale=1.25]{Images/FinishedModelInfected.png}}
	\caption{Результаты работы моделей по динамике инфицированных. Реальные данные, SEIFDR и мультиагентная модели показаны соответственно черным, фиолетовым и зеленым цветами.} 
	\label{FinishedModelInfected}
\end{figure}
\begin{figure}[H]
	\centering{\includegraphics[scale=1.25]{Images/FinishedModelDead.png}}
	\caption{Результаты работы моделей по динамике умерших. Реальные данные, SEIFDR и мультиагентная модели показаны соответственно черным, фиолетовым и зеленым цветами.}
	\label{FinishedModelDead}
\end{figure}


Таблица \ref{tab:Marks:Infected}  показывает аналитические характеристики работы моделей. Заметно, что SEIFDR  модель показывает результат, превосходящий результаты мультиагентной модели.
% Please add the following required packages to your document preamble:
% \usepackage{graphicx}
\begin{table}[H]
	\resizebox{\textwidth}{!}{%
		\begin{tabular}{|l|l|l|}
			\hline
			& SEIFDR                                                                                    & Мультиагентная \\ \hline
			\begin{tabular}[c]{@{}l@{}}Значение целевой функции  (больные)\end{tabular}               & 8.76                                                                                      & 13.905         \\ \hline
			\begin{tabular}[c]{@{}l@{}}Максимальный разброс, кол-во агентов\\ (больные)\end{tabular}    & 28                                                                                        & 40             \\ \hline
			\begin{tabular}[c]{@{}l@{}}Значение целевой функции \ (умершие)\end{tabular}            & 3.54                                                                                    & 5.186          \\ \hline
			\begin{tabular}[c]{@{}l@{}}Максимальный разброс, кол-во агентов \\ \\ (умершие)\end{tabular} & 7                                                                                         & 6              \\ \hline
			Среднее время работы, сек                                                                   & \begin{tabular}[c]{@{}l@{}}10.5\\ \end{tabular} & 45             \\ \hline
		\end{tabular}%
	}
	\caption{Аналитические результаты работы моделей. }
	\label{tab:Marks:Infected}
\end{table}

В ходе калибровки обеих моделей было замечено, что при росте точности прогнозирования значительно возрастает время работы модели, особенно для мультиагентной модели. Таким образом, дальнейшая детализация этой модели приведет к значительному росту затрат по времени при одновременном малом приросте точности моделирования. 




	\newpage
\parindent=1cm %красная строка
\addcontentsline{toc}{section}{Заключение} %Убираем номер , даём имя в оглавлении 
\section*{Заключение} %сам текст заголовка 

Выбранные для реализации подходы представляют собой сочетание строгой формализации с одной стороны и гибкость, позволяющую обходить ограничения и трудности, встречающиеся в других моделях, с другой стороны. 

Согласно результатам таблицы \ref{tab:Marks:Infected} и рисунков \ref{FinishedModelInfected}, \ref{FinishedModelDead}, можно утверждать, что каждая модель обладает своим рядом преимуществ и недостатков: 
\begin{itemize}
	\item SEIFDR модель обладает  повышенной точностью и меньшим временем вычислений (в сравнении с мультиагентной моделью), тем не менее, для её калибровки потребовалось около 8000 итераций, что суммарно привело к  примерно 22 часам настройки. Значительным недостатком данной модели является невозможность быстро добавить к ней параметры, т.к. это приведет к необходимости перерасчета всех формул и, следовательно, к новой калибровке. Также, данная модель (на данном этапе) не позволяет учитывать социальные и географические процессы, которой могут происходить в моделируемом пространстве;
	\item Мультиагентная модель, несмотря на большее значение целевой функции и намного большее время работы, потребовала для калибровки примерно 1400 итераций, т.е. около 17 часов. Следует отметить, что агентный подход имеет значительный потенциал, т.к. позволяет динамично менять модель по мере детализации предметной области и получения большего количества данных, а также учитывает стохастическую природу реальных эпидемических процессов. Значительный <<разлёт>> результатов целевой функции для SEIFDR и мультиагентной моделей в данной работе может быть связан с недостаточным уточнением предметной области или неверным предположением о динамике перемещения агентов. Как и было сказано выше, данный подход имеет большой потенциал и необходимо продолжать работу с использованием различных предположений для получения оптимального результата.
\end{itemize}
В целом, полученные результаты совпадают с общими свойствами соответствующих моделей из первой главы данной выпускной квалификационной работы.


Обе построенные модели позволяют прогнозировать врачу-специалисту эпидемическую обстановку на основе имеющихся у него статистических данных, а также оценить количество ресурсов, необходимых для борьбы со вспышкой инфекции, если такая будет иметь место. Естественным ограничением является территориальный масштаб данных моделей, ограничиваемый одним городом или крупным районом мегаполиса.

Были решены следующие задачи:
\begin{itemize}
	\item Проведен анализ существующих подходов к моделированию распространения инфекционных заболеваний. На основе анализа были выбраны две самые перспективные по мнению автора работы модели;
	\item было выбрано и проанализировано заболевание, которое будет моделироваться, была найдена подробная характеристика по нему;
	\item были выбраны средства моделирования;
	\item было реализовано две модели, затем их точность была увеличена, были проведены стохастические и оптимизационные эксперименты с целью оптимизации параметров моделей;
	\item получены приложения, позволяющие прогнозировать специалисту возможные вспышки заболевания;
	\item построенные модели после обучения были протестированы на другом городе, что подтверждает правильность выбранных подходов и параметров моделей;
	\item приложение для SEIFDR модели имеет готовый простой интерфейс пользователя, т.е. полностью готово для использования специалистом. Интерфейс приложения для мультиагентной модели допускает  улучшения, особенно в области эргономики и оптимального дизайна.
\end{itemize}

Очевидно, что данные модели не могут описать всех факторов, формирующих эпидемиологическую динамику (социальные отношения, меры по защите здоровья, влияние географических и климатических факторов), но даже этого достаточно для выявления общих закономерностей при вспышке лихорадки Эбола и примерной оценки ущерба и ресурсов, необходимых для борьбы с ней. Отдельно заметим, что модели, особенно мультиагентная, допускают гибкое модифицирование в будущем. 
	\newpage
\parindent=1cm %красная строка
\addcontentsline{toc}{section}{Список литературы} %Убираем номер , даём имя в оглавлении 

%http://fkn.ktu10.com/?q=node/6860
\begin{thebibliography}{}
	\bibitem{Kondratiev:2010}  Кондратьев М. А.,	Ивановский Р. И.,	Цыбалова Л.М. Применение агентного подхода к имитационному моделированию процесса распространения заболевания / М.А. Кондратьев, Ивановский Р.И., Цыбалова Л.М. // Научно-технические ведомости СПбГПУ. Физико-математические науки. 2010. № 2. - С. 189–194.
	\bibitem{Bratus:2010} Братусь А.С., Новожилов А.С., Платонов А.П. Динамические системы и модели в биологии /   А.С. Братусь , А.С. Новожилов ,  А.П. Платонов. - М.: Физматлит, 2010. -  400 с.
	\bibitem{Klimentiev:2013} Климентьев К.Е. Компьютерные вирусы и антивирусы: взгляд программиста / К.Е. Климентьев - М.: ДМК-Пресс, 2013. - 656 с.
	\bibitem{Smith:Zombies} When Zombies Attack: Mathematical Modelling of an Outbreak of Zombie Infection / R.Smith, P. Munz, I. Hudea, J. Imad // Infectious Disease Modelling Research Progress. Nova Science Publishers, Inc.
	\bibitem{Klimentiev:2016} Климентьев К.Е. Моделирование влияния подвижных агентов на развитие эпидемий  в сетях <<геометрического>> вида / К.Е. Климентьев  // Известия Самарского научного центра Российской академии наук, т. 18, № 4(4), 2016. - С. 744-748.
	\bibitem{Klimentiev:2015}
	Климентьев К.Е. Случайные графы как модель среды распространения и взаимодействия саморазмножающихся объектов / К.Е. Климентьев  // Известия Самарского научного центра Российской академии наук, т. 17, № 2(5), 2015. - С. 1021 - 1025.
	\bibitem{DiestelR:2005} Diestel, Reinhard. Graph Theory (3rd ed.) // Berlin, New York: Springer-Verlag. 2005.
	\bibitem{Hogan:2017} Hogan, B., Carrasco, J., \& Wellman, B. Visualizing personal networks: Working with participant aided sociograms / B. Hogan, J. Carrasco, B. Wellman // Field Methods, 19(2).  Pp. 116-144.
	\bibitem{Piterson:1981}
	Питерсон Дж. Теория сетей Петри и моделирование систем / Перевод с английского М.В. Горбатовой, В.Л. Торхова, В.Н. Четверикова под редакцией В.А. Горбатова. - М: Мир, 1984. - С. 262.
	\bibitem{Klimentiev:2012} Климентьев К.Е. Применение ГИС-технологий при исследовании распространения
	вредоносных программ/ К.Е. Климентьев  // В сб. «Геоинформационные технологи в проектировании и создании корпоративных
	информационных систем. Межвузовский научн. сборник». – Уфа: изд-во УГАТУ. – 2012. – С. 130-133. 
	\bibitem{Rhee:2007} Rhee, Injong; Shin, Minsu; Hong, Seongik; Lee, Kyunghan; Chong, Song. On the Levy-walk nature of human mobility: Do humans walk like monkeys? / Injong Rhee, Minsu Shin, Seongik Hong, Kyunghan Lee and Song Chong // TIEEE/ACM Transaction on Networking, Vol. 20.2007 - pp. 630-643.
	\bibitem{Privalov:2015} Привалов А.Ю, Царёв А.А. Моделирование передвижений узлов DTN сети с использованием принципа наименьшего действия при выборе локации посещения / А.Ю. Привалов, А.А. Царёв // Самарский государственный университет им. академика С.П. Королева (национальный исследовательский университет). 2015.
	\bibitem{Utakaeva_disser:2011}
	Утакаева И. X. Математические модели	инфекционной динамики	на основе	предфрактальных графов: автореф. дис.  канд. физ-мат. наук.  ФГБОУ ВПО «Северо-Кавказская	государственная гуманитарно-технологическая академия», Ставрополь, 2011.
	\bibitem{Reznikov:2010}
	Резников А.В. Распознавание предфрактальных графов с затравкой, удовлетворяющих условию Оре. / А.В. Резников // Вестник Адыгейского государственного университета. Серия 4: Естественно-математические и технические науки. - 2010.
	\bibitem{Naimanova:2007}  Найманова И.Х., Кочкаров А.М. Об одной задаче распознавания предфрактального графа / И.Х. Найманова, А.М. Кочкаров // Вестник Самарского государственного технического университета. - 2007. - № 1 . - С. 194-196.
	\bibitem{Utukaeva:2008}
	Утакаева И.Х. Алгоритм распознавания предфрактального графа с затравкой регулярной степени / И.Х. Утакаева // Обозрение прикладной и промышленной математики. - 2008. -Том 15.- Выпуск3. - С. 531-533.
	\bibitem{Utukaeva:2011}
	Утакаева И.Х., Кочкаров А.М. Моделирование процесса распространения эпидемии и нахождения возможных очагов заражения на предфрактальном графе / И.Х. Утакаева, А.М. Кочкаров // Сборник трудов 111-ей Всероссийской научно-практической конференции «Перспективные системы и задачи управления». - Таганрог: Издательство Таганрогского технологического института ЮФУ, 2011. - С.273-283. 
	\bibitem{Reznikov_disser:2013} Резников А.В. Исследование свойств и распознавание предфрактальных графов: автореф. дис.  канд. физ-мат. наук.  ФГБОУ ВПО «Северо-Кавказская	государственная гуманитарно-технологическая академия», Ярославль, 2013.
	\bibitem{Bajaramukova:2014} Байрамукова З.Х.,	Кочкаров А.М.,	Кунижева Л.А. Оценка диаметра области распространения вирусов по моделям на предфрактальных графах /   З.Х. Байрамукова ,	А.М. Кочкаров , Л.А. Кунижева //	Научный журнал КубГАУ. - 2014. - № 103(09). - С. 1-10.
	\bibitem{Wolfram_MW:SIR} Kermack-McKendrick Model. Wolfram MathWorld [Электронный ресурс] / Режим доступа: http://mathworld.wolfram.com/Kermack-McKendrickModel.html
	\bibitem{Anderson_May:1979} Anderson, R. M. and May, R. M. "Population Biology of Infectious Diseases: Part I." Nature 280, 361-367, 1979.
	\bibitem{Kermack_McKendrick:1927} Kermack, W. O. and McKendrick, A. G. "A Contribution to the Mathematical Theory of Epidemics." Proc. Roy. Soc. Lond. A 115, 700-721, 1927.
	\bibitem{Bykova:2015} Быкова Ю.С. Мультиагентный подход в имитационном моделировании распространения эпидемии: диплом. работа. Санкт-Петербургский государственный университет, Санкт-Петербург, 2015. 
	\bibitem{Plos_Outbreak:1} Modeling the Impact of Interventions on an Epidemic of Ebola in Sierra Leone and Liberia. Plos.org [Электронный ресурс] / http://currents.plos.org/outbreaks/article/obk-14-0043-modeling-the-impact-of-interventions-on-an-epidemic-of-ebola-in-sierra-leone-and-liberia/
	\bibitem{Plos_Outbreak:2} Modeling the Impact of Interventions on an Epidemic of Ebola in Sierra Leone and Liberia (revision). Plos.org [Электронный ресурс] / http://currents.plos.org/outbreaks/article/modeling-the-impact-of-interventions-on-an-epidemic-of-ebola-in-sierra-leone-and-liberia/
	\bibitem{CDC:2014:1} CDC Telebriefing on Ebola outbreak in West Africa. Centers for Disease Control and Prevention [Электронный ресурс] / Режим доступа: https://www.cdc.gov/media/releases/2014/t0728-ebola.html
	\bibitem{WHO:2018:1} Ebola virus disease. WHO [Электронный ресурс] / Режим доступа: https://www.who.int/en/news-room/fact-sheets/detail/ebola-virus-disease
	\bibitem{Sunit:2014}
	Sunit K.; Ruzek, Daniel, eds. Viral hemorrhagic fevers. Boca Raton: CRC Press, Taylor \& Francis Group. - 2014.
	\bibitem{ncbi:2011} Ebola haemorrhagic fever. NCBI [Электронный ресурс] / Режим доступа: https://www.ncbi.nlm.nih.gov/pmc/articles/PMC3406178/
	\bibitem{ovid:2016} Ebola virus disease and the eye. OVID.com [Электронный ресурс] / Режим доступа: https://insights.ovid.com/crossref?an=00055735-201611000-00011https://insights.ovid.com/crossref?an=00055735-201611000-00011
	\bibitem{Springer:2015} Gastrointestinal and Hepatic Manifestations of Ebola Virus Infection. SPringer.com [Электронный ресурс] / Режим доступа: https://link.springer.com/article/10.1007\%2Fs10620-015-3691-z
	\bibitem{WHOReport:25apr2019} Ebola virus disease – Democratic Republic of the Congo. WHO [Электронный ресурс] / Режим доступа: https://www.who.int/csr/don/25-april-2019-ebola-drc/en/
	\bibitem{WHOReport:02may2019} Ebola virus disease – Democratic Republic of the Congo. WHO [Электронный ресурс] / Режим доступа: https://www.who.int/csr/don/02-may-2019-ebola-drc/en/
	\bibitem{WHOReport:09may2019} Ebola virus disease – Democratic Republic of the Congo. WHO [Электронный ресурс] / Режим доступа: https://www.who.int/csr/don/09-may-2019-ebola-drc/en/
	\bibitem{github_ebola_data:2014} Data for the 2014 ebola outbeak in West Africa. GitHub [Электронный ресурс] / Режим доступа: https://github.com/cmrivers/ebola/
	\bibitem{CDC:Signs} Signs and Symptoms. Centers for Disease Control and Prevention [Электронный ресурс] / Режим доступа: https://www.cdc.gov/vhf/ebola/symptoms/index.html
	\bibitem{TheJournalOfMedicine:2014} Ebola virus disease: a review on epidemiology,
	symptoms, treatment and pathogenesis. Njmonline.nl  [Электронный ресурс] / Режим доступа: https://web.archive.org/web/20141129144852/http://www.njmonline.nl/getpdf.php?t=a\&id=10001148
	\bibitem{ncbi:2014} On the Quarantine Period for Ebola Virus. NCBI [Электронный ресурс] / Режим доступа: https://www.ncbi.nlm.nih.gov/pmc/articles/PMC4205154/
	\bibitem{NYTM:2014} Ask Well: How Does Ebola Spread? How Long Can the Virus Survive? NYTM  [Электронный ресурс] / Режим доступа: https://well.blogs.nytimes.com/2014/10/03/ebola-ask-well-spread-public-transit/	
	\bibitem{CDC:Transmission} Ebola (Ebola Virus Disease), Transmission. Centers for Disease Control and Prevention   [Электронный ресурс] / Режим доступа: https://www.cdc.gov/vhf/ebola/transmission/index.html
	\bibitem{Hunter:2013} Hunter's Tropical Medicine and Emerging Infectious Disease. Google Boocks [Электронный ресурс] / Режим доступа: https://bit.ly/2Hz71VE
	\bibitem{WHO:semen:2016} Interim advice on the sexual transmission of the Ebola virus disease. WHO [Электронный ресурс] / Режим доступа: https://www.who.int/reproductivehealth/topics/rtis/ebola-virus-semen/en/
	\bibitem{ScienceTime:2015} Ebola Can Be Transmitted Through Sex. The Science Time [Электронный ресурс] / Режим доступа: http://www.sciencetimes.com/articles/6000/20150502/ebola-transmitted-through-sex.htm 
	\bibitem{MicroBio:2014} The 2014 Ebola virus disease outbreak in West Africa. Microbiology Society [Электронный ресурс] / Режим доступа:  https://jgv.microbiologyresearch.org/content/journal/jgv/10.1099/vir.0.067199-0
	\bibitem{NetworkX:git} Software for complex networks. GitHub [Электронный ресурс] / Режим доступа: https://networkx.github.io/
	\bibitem{PiP:Wolfram} A Python library with various tools to interact with the Wolfram Language and the Wolfram Cloud. PyPi.org [Электронный ресурс] / Режим доступа: https://pypi.org/project/wolframclient/
	\bibitem{Parahonsky:2011} Старение иммунной системы. Applied Research  [Электронный ресурс] / Режим доступа: https://applied-research.ru/pdf/2011/06/2011\_06\_047.pdf
\end{thebibliography} %список литературы
	\newpage
	
\end{document}