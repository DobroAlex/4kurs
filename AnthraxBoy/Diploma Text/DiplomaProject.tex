\documentclass[a4paper,14pt,russian]{extreport}	%A4 бумага, 14 кегль, русский язык 
\usepackage{extsizes}
\usepackage[onehalfspacing]{setspace} % поулторный интервал %https://proft.me/2013/06/9/latex-ukazanie-mezhstrochnogo-intervala

\usepackage{cmap} % для кодировки шрифтов в pdf
\usepackage[T2A]{fontenc}
%\usepackage{pscyr}
%\usepackage{graphicx} % для вставки картинок
\usepackage{mathptmx} %поддержка textbf
\usepackage{makecell}
\usepackage{textcomp}
\usepackage{multirow} % улучшенное форматирование таблиц
\usepackage{ulem} % подчеркивания

%полужирный шрифт http://tostudents.ru/2009/12/08/smena-shriftov-v-latex-tekst-i-formuly/
\renewcommand{\rmdefault}{ftm} % Times New Roman
\usepackage[utf8]{inputenc}%включаем свою кодировку: koi8-r или utf8 в UNIX, cp1251 в Windows
%\usepackage[]{babel}	%больше поддержки русского языка 
\usepackage[english,russian, russianb]{babel}%используем русский и английский языки с переносами
\usepackage{amssymb,amsfonts,amsmath,mathtext,cite,enumerate,float} %подключаем нужные пакеты расширений

\usepackage[pdftex]{graphicx} %хотим вставлять в диплом рисунки?
\usepackage{cmap} % Улучшенный поиск русских слов в полученном pdf-файле
%\graphicspath{{images/}}%путь к рисункам
\usepackage{fancyhdr}%оформление нумерации 
\usepackage{tableof} %поддержка табличек
\usepackage{mathptmx}%
\usepackage{anyfontsize}% http://texblog.org/2012/08/29/changing-the-font-size-in-latex/
\usepackage{t1enc}%
\usepackage{cite}
\usepackage{graphicx}
\graphicspath{{Images/}} %http://dkhramov.dp.ua/Comp.TexIncludeGraphics#.WwLdJa3sTMU
\DeclareGraphicsExtensions{.pdf,.png,.jpg} %http://dkhramov.dp.ua/Comp.TexIncludeGraphics#.WwLdJa3sTMU
 %https://tex.stackexchange.com/questions/17734/cannot-determine-size-of-graphic
\makeatletter
\renewcommand{\@biblabel}[1]{#1.} % Заменяем библиографию с квадратных скобок на точку:
\makeatother

\usepackage{geometry} % Меняем поля страницы
%QUEST: 3cm или 2 ? Ес\usepackage{cmap} % Улучшенный поиск русских слов в полученном pdf-файлели 3, придется менять форматирование заголовка 
\geometry{left=3cm}% левое поле
\geometry{right=2cm}% правое поле
\geometry{top=2cm}% верхнее поле
\geometry{bottom=2cm}% нижнее поле


\renewcommand{\theenumi}{\arabic{enumi}}% Меняем везде перечисления на цифра.цифра
\renewcommand{\labelenumi}{\arabic{enumi}}% Меняем везде перечисления на цифра.цифра
\renewcommand{\theenumii}{.\arabic{enumii}}% Меняем везде перечисления на цифра.цифра
\renewcommand{\labelenumii}{\arabic{enumi}.\arabic{enumii}.}% Меняем везде перечисления на цифра.цифра
\renewcommand{\theenumiii}{.\arabic{enumiii}}% Меняем везде перечисления на цифра.цифра
\renewcommand{\labelenumiii}{\arabic{enumi}.\arabic{enumii}.\arabic{enumiii}.}% Меняем везде перечисления на цифра.цифра
\addto\captionsrussian{\def\refname{Список используемой литературы}}
\renewcommand{\rmdefault}{ftm}
%NB: три команды ниже переопределяют некотрые шрифты  и дают поддержку жирного и прочиах текстов https://www.linux.org.ru/forum/general/4219163
\renewcommand{\rmdefault}{cmr} % Шрифт с засечками
\renewcommand{\sfdefault}{cmss} % Шрифт без засечек
\renewcommand{\ttdefault}{cmtt} % Моноширинный шрифт
\renewcommand*\thesection{\arabic{section}}
    
\begin{document}
	\renewcommand{\bibname}{Список использованной литературы}
	%\pagestyle{empty} % нумерация выкл.
	    \begin{titlepage}
    \newpage
	\pagestyle{empty} % нумерация выкл.
    \begin{center}
    
	
	{\fontsize{13}{15.6}\selectfont МИНИСТЕРСТВО ОБРАЗОВАНИЯ И НАУКИ РОССИЙСКОЙ ФЕДЕРАЦИИ}\\ 
    \normalsize  {Федеральное государственное автономное образовательное учреждение высшего образования} \\
    
    \large \textbf{<<Крымский  федеральный  университет имени В. И. Вернадского>>} \\  \vspace{2mm}
    (ФГАОУ ВО «КФУ им. В. И. Вернадского»)\\
    
    \textbf{Таврическая академия (структурное подразделение ) \\
    \vspace{2mm}
    Факультет математики и информатики} \\
    \vspace{2mm}
    Кафедра прикладной математики 
    \end{center}
    \vspace{1em}

    \begin{center}
	\normalsize Консманов Алексей Витальевич \\
    \LARGE \textbf{Моделирование распространения инфекционных заболеваний} \\
    \vspace{1em}
    \normalsize Выпускная квалификационная работа 
    \end{center}

    \vspace{1em}
    Обучающегося \hspace*{3cm} \underline{4} курса 
    
    
    Направления подготовки\hspace*{8mm} \underline{01.03.04. Прикладная математика}
    
        
    Форма обучения\hspace*{2.75cm} \underline{очная}\\
    
    
    Научный руководитель
    
    доцент кафедры прикладной математики, 
    
    кандидат физико-математических наук \hspace*{1cm} Ю.Ю. Дюличева
    %\begin{center}
    %	\begin{tabbing}	%http://www.intuit.ru/studies/courses/1137/137/lecture/3835%3Fpage%3D5
    %		\hspace{3cm}Обучающегося \hspace{3cm} \textbf{3 курса}\\ %Быдлокод?
    %		\hspace{2.7cm}Направления подготовки \hspace*{5mm}  \textbf{01.03.04}\\
    %		\hspace{3cm}Форма обучения \hspace{26mm} \textbf{очная}
    %	\end{tabbing}
    
%	\vspace {3em}
%    \flushleft Научный руководитель \hspace{20mm}  старший преподаватель 
    
%    \hspace{75mm}кафедры прикладной математики  
    
    
%    \hspace{75mm}В. А. Лушников
%	\end{center}
    \vspace{\fill}

    \begin{center}
    Симферополь 2019
    \end{center}

    \end{titlepage}% это титульный лист

	\tableofcontents % это оглавление, которое генерируется автоматически
	\thispagestyle{empty}%отключает нумерование страниц до введения включительно 
	%\addcontentsline{toc}{section}{Введение}% будет костыльно выглядеть
	%\newpage
\parindent=1cm %красная строка? 
\begin{center}
	\addcontentsline{toc}{section}{Введение} %Убираем номер , даём имя в оглавлении 
	\section*{Введение} %сам текст заголовка 
	\pagestyle{plain} % нумерация выкл.
	\setcounter{page}{3} % начать нумерацию с номера три
\end{center}


Несмотря на  значительные достижения в области борьбы с инфекционными заболеваниям и порождаемыми ими эпидемиями, данная проблема и связанные с ней задачи все ещё актуальны. Одной из задач, возникающих в связи с борьбой с инфекционными заболеваниями, является исследование путей и способов передачи инфекций. В данной области возникает задача прогнозирования распространения конкретного инфекционного заболевания в некоторой заданной среде, то есть задача построения математической модели, описывающей скорость и масштабы распространения заболевания и его последствия: количество инфицированных и умерших, пространственные масштабы заражения, затраты средств и ресурсов,  необходимые для изоляции больных и последующего излечения. 

Математическое моделирование является мощным и гибким инструментом для исследования реальных процессов и объектов, а также связей между ними. Математическое моделирование удобно применять в тех ситуациях, когда проведение эксперимента с реальными субъектами затруднено по любой причине, что делает его подходящим способом исследования распространения инфекционных заболеваний. Заметим, что учитывая специфическую природу заболеваний и путей их передачи, которые будут изучены и исследованы далее,  важным фактором для проверки адекватности построенной модели является наличие статистической информации, позволяющей тестировать соответствие модели реальным прецедентам.

Важно заметить, что модели, разработанные в данной области, не являются идеальными и универсальными. Основные допущения, принимаемые при построении модели, и недостатки этих моделей будут подробно рассмотрены далее. 

Целью данной работы является анализ существующих моделей распространения инфекционных заболеваний и  создание нескольких различных собственных моделей, позволяющих прогнозировать распространение конкретных заболеваний на основе существующих статистических данных, реализовать полноценное приложение для использования \newline специалистами-эпидемиологами.  Для этого необходимо решить нижеизложенный комплекс задач:
\begin{itemize}
	\item Проанализировать существующие подходы к моделированию, изучить сильные и слабы стороны этих подходов, принимаемые допущения. На основании этого анализа выбрать подход, который будет использоваться при построении собственной модели.
	
	\item выбрать заболевание для тестирования, подробно изучить его и собрать статистические данные о нем, найти статистику, позволяющую сравнить результаты моделирования с реальными;
	
	\item выбрать среду моделирования;
	
	\item разработать и реализовать базовую модель;
	
	\item на основе базовой модели создать модель повышенной точности и детализации;
	
	\item итеративно тестировать и улучшать результаты модели повышенной точности; 
	
	\item разработать отдельное приложение для ввода данных в модель и визуализации полученных моделированием результатов для конечного пользователя -- специалиста-медика или эпидемиолога. 	
\end{itemize}

Для решения поставленного комплекса задач использовались методы математической статистики и теории вероятности, дискретной математики, математического анализа. Разработанные модели основываются на методах агентного / мультиагентного моделирования, их реализации опираются на объектно-ориентированное программирование и современные средства  программного анализа и визуализации данных. 

Объект исследования: изучение распространения инфекционных заболеваний и соответствующих прогностических моделей.

Предмет исследования: построение и анализ моделей распространения инфекционных заболеваний.

Практическая ценность результатов работы: проведено исследование существующих подходов к моделированию,	разработана модель  и программный продукт, реализующий эту модель, также разработан программный продукт для ввода данных в модель специалистом и динамической визуализации результатов моделирования. Разработанные приложения позволяют дать количественный и качественный прогноз распространения инфекционного заболевания для любых  достаточно точно и полно описанных популяции и заболевания.

В первой главе данной выпускной квалификационной работы рассматриваются и анализируются существующие подходы к моделированию распространения инфекционных заболеваний: агентный и мультиагентный подход, алгоритмы на основе случайных  и предфрактальных графов, детерминистический подход на основе дифференциальных уравнений; подводятся итоги и проводится сравнение с выбором наиболее перспективных подходов.

Во второй главе обосновывается выбор целевого заболевания для исследования, приводится подробное описание тех его особенностей, которые необходимы для построения моделей; проводится поиск статистики; проводится анализ средств моделирования с выбором оптимальных; описывается построение моделей, проводившиеся эксперименты, процесс повышения точности модели  и сравнение результатов работы моделей.
 %введение
	%\input{1-SecrecyOfCorrespondence}
	%\input{2-DigitalThreat}
	%\input{3-DataProtection}
	%\newpage
\parindent=1cm %красная строка
\addcontentsline{toc}{section}{Заключение} %Убираем номер , даём имя в оглавлении 
\section*{Заключение} %сам текст заголовка 

Выбранные для реализации подходы представляют собой сочетание строгой формализации с одной стороны и гибкость, позволяющую обходить ограничения и трудности, встречающиеся в других моделях, с другой стороны. 

Согласно результатам таблицы \ref{tab:Marks:Infected} и рисунков \ref{FinishedModelInfected}, \ref{FinishedModelDead}, можно утверждать, что каждая модель обладает своим рядом преимуществ и недостатков: 
\begin{itemize}
	\item SEIFDR модель обладает  повышенной точностью и меньшим временем вычислений (в сравнении с мультиагентной моделью), тем не менее, для её калибровки потребовалось около 8000 итераций, что суммарно привело к  примерно 22 часам настройки. Значительным недостатком данной модели является невозможность быстро добавить к ней параметры, т.к. это приведет к необходимости перерасчета всех формул и, следовательно, к новой калибровке. Также, данная модель (на данном этапе) не позволяет учитывать социальные и географические процессы, которой могут происходить в моделируемом пространстве;
	\item Мультиагентная модель, несмотря на большее значение целевой функции и намного большее время работы, потребовала для калибровки примерно 1400 итераций, т.е. около 17 часов. Следует отметить, что агентный подход имеет значительный потенциал, т.к. позволяет динамично менять модель по мере детализации предметной области и получения большего количества данных, а также учитывает стохастическую природу реальных эпидемических процессов. Значительный <<разлёт>> результатов целевой функции для SEIFDR и мультиагентной моделей в данной работе может быть связан с недостаточным уточнением предметной области или неверным предположением о динамике перемещения агентов. Как и было сказано выше, данный подход имеет большой потенциал и необходимо продолжать работу с использованием различных предположений для получения оптимального результата.
\end{itemize}
В целом, полученные результаты совпадают с общими свойствами соответствующих моделей из первой главы данной выпускной квалификационной работы.


Обе построенные модели позволяют прогнозировать врачу-специалисту эпидемическую обстановку на основе имеющихся у него статистических данных, а также оценить количество ресурсов, необходимых для борьбы со вспышкой инфекции, если такая будет иметь место. Естественным ограничением является территориальный масштаб данных моделей, ограничиваемый одним городом или крупным районом мегаполиса.

Были решены следующие задачи:
\begin{itemize}
	\item Проведен анализ существующих подходов к моделированию распространения инфекционных заболеваний. На основе анализа были выбраны две самые перспективные по мнению автора работы модели;
	\item было выбрано и проанализировано заболевание, которое будет моделироваться, была найдена подробная характеристика по нему;
	\item были выбраны средства моделирования;
	\item было реализовано две модели, затем их точность была увеличена, были проведены стохастические и оптимизационные эксперименты с целью оптимизации параметров моделей;
	\item получены приложения, позволяющие прогнозировать специалисту возможные вспышки заболевания;
	\item построенные модели после обучения были протестированы на другом городе, что подтверждает правильность выбранных подходов и параметров моделей;
	\item приложение для SEIFDR модели имеет готовый простой интерфейс пользователя, т.е. полностью готово для использования специалистом. Интерфейс приложения для мультиагентной модели допускает  улучшения, особенно в области эргономики и оптимального дизайна.
\end{itemize}

Очевидно, что данные модели не могут описать всех факторов, формирующих эпидемиологическую динамику (социальные отношения, меры по защите здоровья, влияние географических и климатических факторов), но даже этого достаточно для выявления общих закономерностей при вспышке лихорадки Эбола и примерной оценки ущерба и ресурсов, необходимых для борьбы с ней. Отдельно заметим, что модели, особенно мультиагентная, допускают гибкое модифицирование в будущем. 
	%\newpage
\parindent=1cm %красная строка
\addcontentsline{toc}{section}{Список литературы} %Убираем номер , даём имя в оглавлении 

%http://fkn.ktu10.com/?q=node/6860
\begin{thebibliography}{}
	\bibitem{Kondratiev:2010}  Кондратьев М. А.,	Ивановский Р. И.,	Цыбалова Л.М. Применение агентного подхода к имитационному моделированию процесса распространения заболевания / М.А. Кондратьев, Ивановский Р.И., Цыбалова Л.М. // Научно-технические ведомости СПбГПУ. Физико-математические науки. 2010. № 2. - С. 189–194.
	\bibitem{Bratus:2010} Братусь А.С., Новожилов А.С., Платонов А.П. Динамические системы и модели в биологии /   А.С. Братусь , А.С. Новожилов ,  А.П. Платонов. - М.: Физматлит, 2010. -  400 с.
	\bibitem{Klimentiev:2013} Климентьев К.Е. Компьютерные вирусы и антивирусы: взгляд программиста / К.Е. Климентьев - М.: ДМК-Пресс, 2013. - 656 с.
	\bibitem{Smith:Zombies} When Zombies Attack: Mathematical Modelling of an Outbreak of Zombie Infection / R.Smith, P. Munz, I. Hudea, J. Imad // Infectious Disease Modelling Research Progress. Nova Science Publishers, Inc.
	\bibitem{Klimentiev:2016} Климентьев К.Е. Моделирование влияния подвижных агентов на развитие эпидемий  в сетях <<геометрического>> вида / К.Е. Климентьев  // Известия Самарского научного центра Российской академии наук, т. 18, № 4(4), 2016. - С. 744-748.
	\bibitem{Klimentiev:2015}
	Климентьев К.Е. Случайные графы как модель среды распространения и взаимодействия саморазмножающихся объектов / К.Е. Климентьев  // Известия Самарского научного центра Российской академии наук, т. 17, № 2(5), 2015. - С. 1021 - 1025.
	\bibitem{DiestelR:2005} Diestel, Reinhard. Graph Theory (3rd ed.) // Berlin, New York: Springer-Verlag. 2005.
	\bibitem{Hogan:2017} Hogan, B., Carrasco, J., \& Wellman, B. Visualizing personal networks: Working with participant aided sociograms / B. Hogan, J. Carrasco, B. Wellman // Field Methods, 19(2).  Pp. 116-144.
	\bibitem{Piterson:1981}
	Питерсон Дж. Теория сетей Петри и моделирование систем / Перевод с английского М.В. Горбатовой, В.Л. Торхова, В.Н. Четверикова под редакцией В.А. Горбатова. - М: Мир, 1984. - С. 262.
	\bibitem{Klimentiev:2012} Климентьев К.Е. Применение ГИС-технологий при исследовании распространения
	вредоносных программ/ К.Е. Климентьев  // В сб. «Геоинформационные технологи в проектировании и создании корпоративных
	информационных систем. Межвузовский научн. сборник». – Уфа: изд-во УГАТУ. – 2012. – С. 130-133. 
	\bibitem{Rhee:2007} Rhee, Injong; Shin, Minsu; Hong, Seongik; Lee, Kyunghan; Chong, Song. On the Levy-walk nature of human mobility: Do humans walk like monkeys? / Injong Rhee, Minsu Shin, Seongik Hong, Kyunghan Lee and Song Chong // TIEEE/ACM Transaction on Networking, Vol. 20.2007 - pp. 630-643.
	\bibitem{Privalov:2015} Привалов А.Ю, Царёв А.А. Моделирование передвижений узлов DTN сети с использованием принципа наименьшего действия при выборе локации посещения / А.Ю. Привалов, А.А. Царёв // Самарский государственный университет им. академика С.П. Королева (национальный исследовательский университет). 2015.
	\bibitem{Utakaeva_disser:2011}
	Утакаева И. X. Математические модели	инфекционной динамики	на основе	предфрактальных графов: автореф. дис.  канд. физ-мат. наук.  ФГБОУ ВПО «Северо-Кавказская	государственная гуманитарно-технологическая академия», Ставрополь, 2011.
	\bibitem{Reznikov:2010}
	Резников А.В. Распознавание предфрактальных графов с затравкой, удовлетворяющих условию Оре. / А.В. Резников // Вестник Адыгейского государственного университета. Серия 4: Естественно-математические и технические науки. - 2010.
	\bibitem{Naimanova:2007}  Найманова И.Х., Кочкаров А.М. Об одной задаче распознавания предфрактального графа / И.Х. Найманова, А.М. Кочкаров // Вестник Самарского государственного технического университета. - 2007. - № 1 . - С. 194-196.
	\bibitem{Utukaeva:2008}
	Утакаева И.Х. Алгоритм распознавания предфрактального графа с затравкой регулярной степени / И.Х. Утакаева // Обозрение прикладной и промышленной математики. - 2008. -Том 15.- Выпуск3. - С. 531-533.
	\bibitem{Utukaeva:2011}
	Утакаева И.Х., Кочкаров А.М. Моделирование процесса распространения эпидемии и нахождения возможных очагов заражения на предфрактальном графе / И.Х. Утакаева, А.М. Кочкаров // Сборник трудов 111-ей Всероссийской научно-практической конференции «Перспективные системы и задачи управления». - Таганрог: Издательство Таганрогского технологического института ЮФУ, 2011. - С.273-283. 
	\bibitem{Reznikov_disser:2013} Резников А.В. Исследование свойств и распознавание предфрактальных графов: автореф. дис.  канд. физ-мат. наук.  ФГБОУ ВПО «Северо-Кавказская	государственная гуманитарно-технологическая академия», Ярославль, 2013.
	\bibitem{Bajaramukova:2014} Байрамукова З.Х.,	Кочкаров А.М.,	Кунижева Л.А. Оценка диаметра области распространения вирусов по моделям на предфрактальных графах /   З.Х. Байрамукова ,	А.М. Кочкаров , Л.А. Кунижева //	Научный журнал КубГАУ. - 2014. - № 103(09). - С. 1-10.
	\bibitem{Wolfram_MW:SIR} Kermack-McKendrick Model. Wolfram MathWorld [Электронный ресурс] / Режим доступа: http://mathworld.wolfram.com/Kermack-McKendrickModel.html
	\bibitem{Anderson_May:1979} Anderson, R. M. and May, R. M. "Population Biology of Infectious Diseases: Part I." Nature 280, 361-367, 1979.
	\bibitem{Kermack_McKendrick:1927} Kermack, W. O. and McKendrick, A. G. "A Contribution to the Mathematical Theory of Epidemics." Proc. Roy. Soc. Lond. A 115, 700-721, 1927.
	\bibitem{Bykova:2015} Быкова Ю.С. Мультиагентный подход в имитационном моделировании распространения эпидемии: диплом. работа. Санкт-Петербургский государственный университет, Санкт-Петербург, 2015. 
	\bibitem{Plos_Outbreak:1} Modeling the Impact of Interventions on an Epidemic of Ebola in Sierra Leone and Liberia. Plos.org [Электронный ресурс] / http://currents.plos.org/outbreaks/article/obk-14-0043-modeling-the-impact-of-interventions-on-an-epidemic-of-ebola-in-sierra-leone-and-liberia/
	\bibitem{Plos_Outbreak:2} Modeling the Impact of Interventions on an Epidemic of Ebola in Sierra Leone and Liberia (revision). Plos.org [Электронный ресурс] / http://currents.plos.org/outbreaks/article/modeling-the-impact-of-interventions-on-an-epidemic-of-ebola-in-sierra-leone-and-liberia/
	\bibitem{CDC:2014:1} CDC Telebriefing on Ebola outbreak in West Africa. Centers for Disease Control and Prevention [Электронный ресурс] / Режим доступа: https://www.cdc.gov/media/releases/2014/t0728-ebola.html
	\bibitem{WHO:2018:1} Ebola virus disease. WHO [Электронный ресурс] / Режим доступа: https://www.who.int/en/news-room/fact-sheets/detail/ebola-virus-disease
	\bibitem{Sunit:2014}
	Sunit K.; Ruzek, Daniel, eds. Viral hemorrhagic fevers. Boca Raton: CRC Press, Taylor \& Francis Group. - 2014.
	\bibitem{ncbi:2011} Ebola haemorrhagic fever. NCBI [Электронный ресурс] / Режим доступа: https://www.ncbi.nlm.nih.gov/pmc/articles/PMC3406178/
	\bibitem{ovid:2016} Ebola virus disease and the eye. OVID.com [Электронный ресурс] / Режим доступа: https://insights.ovid.com/crossref?an=00055735-201611000-00011https://insights.ovid.com/crossref?an=00055735-201611000-00011
	\bibitem{Springer:2015} Gastrointestinal and Hepatic Manifestations of Ebola Virus Infection. SPringer.com [Электронный ресурс] / Режим доступа: https://link.springer.com/article/10.1007\%2Fs10620-015-3691-z
	\bibitem{WHOReport:25apr2019} Ebola virus disease – Democratic Republic of the Congo. WHO [Электронный ресурс] / Режим доступа: https://www.who.int/csr/don/25-april-2019-ebola-drc/en/
	\bibitem{WHOReport:02may2019} Ebola virus disease – Democratic Republic of the Congo. WHO [Электронный ресурс] / Режим доступа: https://www.who.int/csr/don/02-may-2019-ebola-drc/en/
	\bibitem{WHOReport:09may2019} Ebola virus disease – Democratic Republic of the Congo. WHO [Электронный ресурс] / Режим доступа: https://www.who.int/csr/don/09-may-2019-ebola-drc/en/
	\bibitem{github_ebola_data:2014} Data for the 2014 ebola outbeak in West Africa. GitHub [Электронный ресурс] / Режим доступа: https://github.com/cmrivers/ebola/
	\bibitem{CDC:Signs} Signs and Symptoms. Centers for Disease Control and Prevention [Электронный ресурс] / Режим доступа: https://www.cdc.gov/vhf/ebola/symptoms/index.html
	\bibitem{TheJournalOfMedicine:2014} Ebola virus disease: a review on epidemiology,
	symptoms, treatment and pathogenesis. Njmonline.nl  [Электронный ресурс] / Режим доступа: https://web.archive.org/web/20141129144852/http://www.njmonline.nl/getpdf.php?t=a\&id=10001148
	\bibitem{ncbi:2014} On the Quarantine Period for Ebola Virus. NCBI [Электронный ресурс] / Режим доступа: https://www.ncbi.nlm.nih.gov/pmc/articles/PMC4205154/
	\bibitem{NYTM:2014} Ask Well: How Does Ebola Spread? How Long Can the Virus Survive? NYTM  [Электронный ресурс] / Режим доступа: https://well.blogs.nytimes.com/2014/10/03/ebola-ask-well-spread-public-transit/	
	\bibitem{CDC:Transmission} Ebola (Ebola Virus Disease), Transmission. Centers for Disease Control and Prevention   [Электронный ресурс] / Режим доступа: https://www.cdc.gov/vhf/ebola/transmission/index.html
	\bibitem{Hunter:2013} Hunter's Tropical Medicine and Emerging Infectious Disease. Google Boocks [Электронный ресурс] / Режим доступа: https://bit.ly/2Hz71VE
	\bibitem{WHO:semen:2016} Interim advice on the sexual transmission of the Ebola virus disease. WHO [Электронный ресурс] / Режим доступа: https://www.who.int/reproductivehealth/topics/rtis/ebola-virus-semen/en/
	\bibitem{ScienceTime:2015} Ebola Can Be Transmitted Through Sex. The Science Time [Электронный ресурс] / Режим доступа: http://www.sciencetimes.com/articles/6000/20150502/ebola-transmitted-through-sex.htm 
	\bibitem{MicroBio:2014} The 2014 Ebola virus disease outbreak in West Africa. Microbiology Society [Электронный ресурс] / Режим доступа:  https://jgv.microbiologyresearch.org/content/journal/jgv/10.1099/vir.0.067199-0
	\bibitem{NetworkX:git} Software for complex networks. GitHub [Электронный ресурс] / Режим доступа: https://networkx.github.io/
	\bibitem{PiP:Wolfram} A Python library with various tools to interact with the Wolfram Language and the Wolfram Cloud. PyPi.org [Электронный ресурс] / Режим доступа: https://pypi.org/project/wolframclient/
	\bibitem{Parahonsky:2011} Старение иммунной системы. Applied Research  [Электронный ресурс] / Режим доступа: https://applied-research.ru/pdf/2011/06/2011\_06\_047.pdf
\end{thebibliography} %список литературы
	\newpage
	
\end{document}