\newpage
\parindent=1cm %красная строка
\addcontentsline{toc}{section}{Список литературы} %Убираем номер , даём имя в оглавлении 

%http://fkn.ktu10.com/?q=node/6860
\begin{thebibliography}{}
	\bibitem{Kondratiev:2010}  Кондратьев М. А.,	Ивановский Р. И.,	Цыбалова Л.М. Применение агентного подхода к имитационному моделированию процесса распространения заболевания / М.А. Кондратьев, Ивановский Р.И., Цыбалова Л.М. // Научно-технические ведомости СПбГПУ. Физико-математические науки. 2010. № 2. - С. 189–194.
	\bibitem{Bratus:2010} Братусь А.С., Новожилов А.С., Платонов А.П. Динамические системы и модели в биологии /   А.С. Братусь , А.С. Новожилов ,  А.П. Платонов. - М.: Физматлит, 2010. -  400 с.
	\bibitem{Klimentiev:2013} Климентьев К.Е. Компьютерные вирусы и антивирусы: взгляд программиста / К.Е. Климентьев - М.: ДМК-Пресс, 2013. - 656 с.
	\bibitem{Smith:Zombies} When Zombies Attack: Mathematical Modelling of an Outbreak of Zombie Infection / R.Smith, P. Munz, I. Hudea, J. Imad // Infectious Disease Modelling Research Progress. Nova Science Publishers, Inc.
	\bibitem{Klimentiev:2016} Климентьев К.Е. Моделирование влияния подвижных агентов на развитие эпидемий  в сетях <<геометрического>> вида / К.Е. Климентьев  // Известия Самарского научного центра Российской академии наук, т. 18, № 4(4), 2016. - С. 744-748.
	\bibitem{Klimentiev:2015}
	Климентьев К.Е. Случайные графы как модель среды распространения и взаимодействия саморазмножающихся объектов / К.Е. Климентьев  // Известия Самарского научного центра Российской академии наук, т. 17, № 2(5), 2015. - С. 1021 - 1025.
	\bibitem{DiestelR:2005} Diestel, Reinhard. Graph Theory (3rd ed.) // Berlin, New York: Springer-Verlag. 2005.
	\bibitem{Hogan:2017} Hogan, B., Carrasco, J., \& Wellman, B. Visualizing personal networks: Working with participant aided sociograms / B. Hogan, J. Carrasco, B. Wellman // Field Methods, 19(2).  Pp. 116-144.
	\bibitem{Piterson:1981}
	Питерсон Дж. Теория сетей Петри и моделирование систем / Перевод с английского М.В. Горбатовой, В.Л. Торхова, В.Н. Четверикова под редакцией В.А. Горбатова. - М: Мир, 1984. - С. 262.
	\bibitem{Klimentiev:2012} Климентьев К.Е. Применение ГИС-технологий при исследовании распространения
	вредоносных программ/ К.Е. Климентьев  // В сб. «Геоинформационные технологи в проектировании и создании корпоративных
	информационных систем. Межвузовский научн. сборник». – Уфа: изд-во УГАТУ. – 2012. – С. 130-133. 
	\bibitem{Rhee:2007} Rhee, Injong; Shin, Minsu; Hong, Seongik; Lee, Kyunghan; Chong, Song. On the Levy-walk nature of human mobility: Do humans walk like monkeys? / Injong Rhee, Minsu Shin, Seongik Hong, Kyunghan Lee and Song Chong // TIEEE/ACM Transaction on Networking, Vol. 20.2007 - pp. 630-643.
	\bibitem{Privalov:2015} Привалов А.Ю, Царёв А.А. Моделирование передвижений узлов DTN сети с использованием принципа наименьшего действия при выборе локации посещения / А.Ю. Привалов, А.А. Царёв // Самарский государственный университет им. академика С.П. Королева (национальный исследовательский университет). 2015.
	\bibitem{Utakaeva_disser:2011}
	Утакаева И. X. Математические модели	инфекционной динамики	на основе	предфрактальных графов: автореф. дис.  канд. физ-мат. наук.  ФГБОУ ВПО «Северо-Кавказская	государственная гуманитарно-технологическая академия», Ставрополь, 2011.
	\bibitem{Reznikov:2010}
	Резников А.В. Распознавание предфрактальных графов с затравкой, удовлетворяющих условию Оре. / А.В. Резников // Вестник Адыгейского государственного университета. Серия 4: Естественно-математические и технические науки. - 2010.
	\bibitem{Naimanova:2007}  Найманова И.Х., Кочкаров А.М. Об одной задаче распознавания предфрактального графа / И.Х. Найманова, А.М. Кочкаров // Вестник Самарского государственного технического университета. - 2007. - № 1 . - С. 194-196.
	\bibitem{Utukaeva:2008}
	Утакаева И.Х. Алгоритм распознавания предфрактального графа с затравкой регулярной степени / И.Х. Утакаева // Обозрение прикладной и промышленной математики. - 2008. -Том 15.- Выпуск3. - С. 531-533.
	\bibitem{Utukaeva:2011}
	Утакаева И.Х., Кочкаров А.М. Моделирование процесса распространения эпидемии и нахождения возможных очагов заражения на предфрактальном графе / И.Х. Утакаева, А.М. Кочкаров // Сборник трудов 111-ей Всероссийской научно-практической конференции «Перспективные системы и задачи управления». - Таганрог: Издательство Таганрогского технологического института ЮФУ, 2011. - С.273-283. 
	\bibitem{Reznikov_disser:2013} Резников А.В. Исследование свойств и распознавание предфрактальных графов: автореф. дис.  канд. физ-мат. наук.  ФГБОУ ВПО «Северо-Кавказская	государственная гуманитарно-технологическая академия», Ярославль, 2013.
	\bibitem{Bajaramukova:2014} Байрамукова З.Х.,	Кочкаров А.М.,	Кунижева Л.А. Оценка диаметра области распространения вирусов по моделям на предфрактальных графах /   З.Х. Байрамукова ,	А.М. Кочкаров , Л.А. Кунижева //	Научный журнал КубГАУ. - 2014. - № 103(09). - С. 1-10.
	\bibitem{Wolfram_MW:SIR} Kermack-McKendrick Model. Wolfram MathWorld [Электронный ресурс] / Режим доступа: http://mathworld.wolfram.com/Kermack-McKendrickModel.html
	\bibitem{Anderson_May:1979} Anderson, R. M. and May, R. M. "Population Biology of Infectious Diseases: Part I." Nature 280, 361-367, 1979.
	\bibitem{Kermack_McKendrick:1927} Kermack, W. O. and McKendrick, A. G. "A Contribution to the Mathematical Theory of Epidemics." Proc. Roy. Soc. Lond. A 115, 700-721, 1927.
	\bibitem{Bykova:2015} Быкова Ю.С. Мультиагентный подход в имитационном моделировании распространения эпидемии: диплом. работа. Санкт-Петербургский государственный университет, Санкт-Петербург, 2015. 
	\bibitem{Plos_Outbreak:1} Modeling the Impact of Interventions on an Epidemic of Ebola in Sierra Leone and Liberia. Plos.org [Электронный ресурс] / http://currents.plos.org/outbreaks/article/obk-14-0043-modeling-the-impact-of-interventions-on-an-epidemic-of-ebola-in-sierra-leone-and-liberia/
	\bibitem{Plos_Outbreak:2} Modeling the Impact of Interventions on an Epidemic of Ebola in Sierra Leone and Liberia (revision). Plos.org [Электронный ресурс] / http://currents.plos.org/outbreaks/article/modeling-the-impact-of-interventions-on-an-epidemic-of-ebola-in-sierra-leone-and-liberia/
	\bibitem{CDC:2014:1} CDC Telebriefing on Ebola outbreak in West Africa. Centers for Disease Control and Prevention [Электронный ресурс] / Режим доступа: https://www.cdc.gov/media/releases/2014/t0728-ebola.html
	\bibitem{WHO:2018:1} Ebola virus disease. WHO [Электронный ресурс] / Режим доступа: https://www.who.int/en/news-room/fact-sheets/detail/ebola-virus-disease
	\bibitem{Sunit:2014}
	Sunit K.; Ruzek, Daniel, eds. Viral hemorrhagic fevers. Boca Raton: CRC Press, Taylor \& Francis Group. - 2014.
	\bibitem{ncbi:2011} Ebola haemorrhagic fever. NCBI [Электронный ресурс] / Режим доступа: https://www.ncbi.nlm.nih.gov/pmc/articles/PMC3406178/
	\bibitem{ovid:2016} Ebola virus disease and the eye. OVID.com [Электронный ресурс] / Режим доступа: https://insights.ovid.com/crossref?an=00055735-201611000-00011https://insights.ovid.com/crossref?an=00055735-201611000-00011
	\bibitem{Springer:2015} Gastrointestinal and Hepatic Manifestations of Ebola Virus Infection. SPringer.com [Электронный ресурс] / Режим доступа: https://link.springer.com/article/10.1007\%2Fs10620-015-3691-z
	\bibitem{WHOReport:25apr2019} Ebola virus disease – Democratic Republic of the Congo. WHO [Электронный ресурс] / Режим доступа: https://www.who.int/csr/don/25-april-2019-ebola-drc/en/
	\bibitem{WHOReport:02may2019} Ebola virus disease – Democratic Republic of the Congo. WHO [Электронный ресурс] / Режим доступа: https://www.who.int/csr/don/02-may-2019-ebola-drc/en/
	\bibitem{WHOReport:09may2019} Ebola virus disease – Democratic Republic of the Congo. WHO [Электронный ресурс] / Режим доступа: https://www.who.int/csr/don/09-may-2019-ebola-drc/en/
	\bibitem{github_ebola_data:2014} Data for the 2014 ebola outbeak in West Africa. GitHub [Электронный ресурс] / Режим доступа: https://github.com/cmrivers/ebola/
	\bibitem{CDC:Signs} Signs and Symptoms. Centers for Disease Control and Prevention [Электронный ресурс] / Режим доступа: https://www.cdc.gov/vhf/ebola/symptoms/index.html
	\bibitem{TheJournalOfMedicine:2014} Ebola virus disease: a review on epidemiology,
	symptoms, treatment and pathogenesis. Njmonline.nl  [Электронный ресурс] / Режим доступа: https://web.archive.org/web/20141129144852/http://www.njmonline.nl/getpdf.php?t=a\&id=10001148
	\bibitem{ncbi:2014} On the Quarantine Period for Ebola Virus. NCBI [Электронный ресурс] / Режим доступа: https://www.ncbi.nlm.nih.gov/pmc/articles/PMC4205154/
	\bibitem{NYTM:2014} Ask Well: How Does Ebola Spread? How Long Can the Virus Survive? NYTM  [Электронный ресурс] / Режим доступа: https://well.blogs.nytimes.com/2014/10/03/ebola-ask-well-spread-public-transit/	
	\bibitem{CDC:Transmission} Ebola (Ebola Virus Disease), Transmission. Centers for Disease Control and Prevention   [Электронный ресурс] / Режим доступа: https://www.cdc.gov/vhf/ebola/transmission/index.html
	\bibitem{Hunter:2013} Hunter's Tropical Medicine and Emerging Infectious Disease. Google Boocks [Электронный ресурс] / Режим доступа: https://bit.ly/2Hz71VE
	\bibitem{WHO:semen:2016} Interim advice on the sexual transmission of the Ebola virus disease. WHO [Электронный ресурс] / Режим доступа: https://www.who.int/reproductivehealth/topics/rtis/ebola-virus-semen/en/
	\bibitem{ScienceTime:2015} Ebola Can Be Transmitted Through Sex. The Science Time [Электронный ресурс] / Режим доступа: http://www.sciencetimes.com/articles/6000/20150502/ebola-transmitted-through-sex.htm 
	\bibitem{MicroBio:2014} The 2014 Ebola virus disease outbreak in West Africa. Microbiology Society [Электронный ресурс] / Режим доступа:  https://jgv.microbiologyresearch.org/content/journal/jgv/10.1099/vir.0.067199-0
	\bibitem{NetworkX:git} Software for complex networks. GitHub [Электронный ресурс] / Режим доступа: https://networkx.github.io/
	\bibitem{PiP:Wolfram} A Python library with various tools to interact with the Wolfram Language and the Wolfram Cloud. PyPi.org [Электронный ресурс] / Режим доступа: https://pypi.org/project/wolframclient/
	\bibitem{Parahonsky:2011} Старение иммунной системы. Applied Research  [Электронный ресурс] / Режим доступа: https://applied-research.ru/pdf/2011/06/2011\_06\_047.pdf
\end{thebibliography}