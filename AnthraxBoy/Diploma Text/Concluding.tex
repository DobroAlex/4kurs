\newpage
\parindent=1cm %красная строка
\addcontentsline{toc}{section}{Заключение} %Убираем номер , даём имя в оглавлении 
\section*{Заключение} %сам текст заголовка 

Выбранные для реализации подходы представляют собой сочетание строгой формализации с одной стороны и гибкость, позволяющую обходить ограничения и трудности, встречающиеся в других моделях, с другой стороны. 

Агентный подход имеет большой потенциал, т.к позволяет динамично менять модель по мере детализации предметной области и получения большего количества данных, а также учитывает стохастическую природу реальных эпидемических процессов.

Обе построенные модели позволяют прогнозировать врачу-специалисту эпидемическую обстановку на основе имеющихся у него статистических данных, а также оценить количество ресурсов, необходимых для борьбы со вспышкой инфекции, если такая будет иметь место. Естественным ограничением является территориальный масштаб данных моделей, ограничиваемый одним городом или крупным районом мегаполиса.

Были решены следующие задачи:
\begin{itemize}
	\item Проведен анализ существующих подходов к моделированию распространения инфекционных заболеваний. На основе анализа были выбраны две самые перспективные по мнению автора работы модели;
	\item было выбрано и проанализировано заболевание, которое будет моделироваться, была найдена подробная характеристика по нему;
	\item были выбраны средства моделирования;
	\item было реализовано две модели, затем их точность была увеличена, были проведены стохастические и оптимизационные эксперименты с целью оптимизации параметров моделей;
	\item получены приложения, позволяющие прогнозировать специалисту возможные вспышки заболевания;
	\item построенные модели после обучения были протестированы на другом городе, что подтверждает правильность выбранных подходов и параметров моделей.
\end{itemize}

Очевидно, что данные модели не могут описать всех факторов, формирующих эпидемиологическую динамику (социальные отношения, меры по защите здоровья, влияние географических и климатических факторов), но даже этого достаточно для выявления общих закономерностей при вспышке лихорадки Эбола и примерной оценки ущерба и ресурсов, необходимых для борьбы с ней. Отдельно заметим, что модели, особенно мультиагентная, допускают гибкое модифицирование в будущем. 