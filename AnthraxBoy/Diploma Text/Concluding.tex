\newpage
\parindent=1cm %красная строка
\addcontentsline{toc}{section}{Заключение} %Убираем номер , даём имя в оглавлении 
\section*{Заключение} %сам текст заголовка 

Выбранные для реализации подходы представляют собой сочетание строгой формализации с одной стороны и гибкость, позволяющую обходить ограничения и трудности, встречающиеся в других моделях, с другой стороны. 

Согласно результатам таблицы \ref{tab:Marks:Infected} и рисунков \ref{FinishedModelInfected}, \ref{FinishedModelDead}, можно утверждать, что каждая модель обладает своим рядом преимуществ и недостатков: 
\begin{itemize}
	\item SEIFDR модель обладает более повышенной точностью и меньшим временем вычислений (в сравнении с мультиагентной моделью), тем не менее, для её калибровки потребовалось около 8000 итераций, что суммарно привело к  примерно 22 часам настройки. Значительным недостатком данной модели является невозможность быстро добавить к ней параметры, т.к. это приведет к необходимости перерасчета всех формул и, следовательно, к новой калибровке. Также, данная модель (на данном этапе) не позволяет учитывать социальные и географические процессы, которой могут происходить в моделируемом пространстве;
	\item Мультиагентная модель, несмотря на большее значение целевой функции и намного большее время работы, потребовала для калибровки примерно 1400 итераций, т.е. около 17 часов. Следует отметить, что агентный подход имеет значительный потенциал, т.к. позволяет динамично менять модель по мере детализации предметной области и получения большего количества данных, а также учитывает стохастическую природу реальных эпидемических процессов. Значительный <<разлёт>> результатов целевой функции для SEIFDR и мультиагентной моделей в данной работе может быть связан с недостаточным уточнением предметной области или неверным предположением о динамике перемещения агентов. Как и было сказано выше, данный подход имеет большой потенциал и необходимо продолжать работу с использованием различных предположений для получения оптимального результата.
\end{itemize}
В целом, полученные результаты совпадают с общими свойствами соответствующих моделей из первой главы данной выпускной квалификационной работы.


Обе построенные модели позволяют прогнозировать врачу-специалисту эпидемическую обстановку на основе имеющихся у него статистических данных, а также оценить количество ресурсов, необходимых для борьбы со вспышкой инфекции, если такая будет иметь место. Естественным ограничением является территориальный масштаб данных моделей, ограничиваемый одним городом или крупным районом мегаполиса.

Были решены следующие задачи:
\begin{itemize}
	\item Проведен анализ существующих подходов к моделированию распространения инфекционных заболеваний. На основе анализа были выбраны две самые перспективные по мнению автора работы модели;
	\item было выбрано и проанализировано заболевание, которое будет моделироваться, была найдена подробная характеристика по нему;
	\item были выбраны средства моделирования;
	\item было реализовано две модели, затем их точность была увеличена, были проведены стохастические и оптимизационные эксперименты с целью оптимизации параметров моделей;
	\item получены приложения, позволяющие прогнозировать специалисту возможные вспышки заболевания;
	\item построенные модели после обучения были протестированы на другом городе, что подтверждает правильность выбранных подходов и параметров моделей;
	\item приложение для SEIFDR модели имеет готовый простой интерфейс пользователя, т.е. полностью готово для использования специалистом. Интерфейс приложения для мультиагентной модели допускает  улучшения, особенно в области эргономики и оптимального дизайна.
\end{itemize}

Очевидно, что данные модели не могут описать всех факторов, формирующих эпидемиологическую динамику (социальные отношения, меры по защите здоровья, влияние географических и климатических факторов), но даже этого достаточно для выявления общих закономерностей при вспышке лихорадки Эбола и примерной оценки ущерба и ресурсов, необходимых для борьбы с ней. Отдельно заметим, что модели, особенно мультиагентная, допускают гибкое модифицирование в будущем. 