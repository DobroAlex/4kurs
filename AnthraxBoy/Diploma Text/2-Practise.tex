\newpage
\parindent=1cm %красная строка
\begin{center}
		
		\section{Реализация компьютерных моделей распространения инфекционных заболеваний}
		
\end{center}


Как уже было указано  в предыдущем разделе, для реализации выбраны мультиагентная и SEIFDR модели. Как указано в целях работы, необходимо выбрать целевое заболевание и среду моделирования.

\subsection{Выбор заболевания, его краткое описание}
Прежде всего, исследуемое заболевание должно обладать ярко выраженными симптомами, что позволяет получить высокорепрезентативную  выборку.


В качества целевого заболевания для анализа выбрана геморрагическая лихорадка Эб$\acute{o}$ла -- острое вирусное высококонтагиозное заболевание, поражающие человека и других приматов, . Данное заболевание характеризуется высокой заразностью (человеческие останки остаются заразными до 50 дней 
%REF: https://www.cdc.gov/media/releases/2014/t0728-ebola.html
), коротким инкубационным периодом, резкими клиническими проявлениями и очень высокой смертностью(до 90\%, в среднем около 50\%). %REF: https://www.who.int/en/news-room/fact-sheets/detail/ebola-virus-disease
%REF: https://books.google.com/books?id=l5MtJdDhie0C&pg=PA444#v=onepage&q&f=false Singh, Sunit K.; Ruzek, Daniel, eds. (2014). Viral hemorrhagic fevers. Boca Raton: CRC Press, Taylor & Francis Group. СТР 444
Инфицирование происходит при прямом контакте через биологические жидкости, выделения, предметы. Основные симптомы (по времени проявления):  усталость, лихорадка, слабость, миалгия,  артралгия, боли в горле, чиханье и диарея,  обширные внутренние и внешние кровотечения, в том числе коагулопатия, кровохарканье, кровь в глазном белке, гиповолемический шок, приводящие к смерти. 
%REF: https://www.ncbi.nlm.nih.gov/pmc/articles/PMC3406178/
%REF: https://insights.ovid.com/crossref?an=00055735-201611000-00011
%REF: https://link.springer.com/article/10.1007%2Fs10620-015-3691-z

Данное заболевание удобно для изучения тем, что распространяется в странах с низким уровнем гигиены и медицинской поддержки, что позволяет изучить <<чистый>> вирус на <<чистой>> популяции, т.е модель, описывающая вирус, может пренебрегать медицинским мероприятиями; также имеется подробная посуточная статистика с низким уровнем шума по государствам и отдельным городам.


Типичные группы риска: медработники, участники похоронных команд или другие лица с непосредственным доступом к трупу, члены семьи и другие лица, находящиеся в тесном контакте с инфицированным, ведущие промысел во влажных экваториальных лесах охотники, при любом контакте с трупами инфицированных животных или при использовании зараженных животных в пищевой промышленности. Также в этой группе находятся все жители Центральной и  Южной (кроме ЮАР) Африки  и Западного и Восточного  побережий -- в этих регионах наблюдается низкий уровень санитарии, небезопасные промыслы на диких животных, низкий уровень медицины и образования людей в целом и неблагоприятная социально-политическая ситуация.


