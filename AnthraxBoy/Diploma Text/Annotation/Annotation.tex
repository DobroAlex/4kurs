\documentclass[a4paper,14pt,russian]{extreport}	%A4 бумага, 14 кегль, русский язык 
\usepackage{extsizes}
\usepackage[onehalfspacing]{setspace}
\usepackage{geometry} % Меняем поля страницы
%QUEST: 3cm или 2 ? Ес\usepackage{cmap} % Улучшенный поиск русских слов в полученном pdf-файлели 3, придется менять форматирование заголовка 
\geometry{left=3cm}% левое поле
\geometry{right=2cm}% правое поле
\geometry{top=2cm}% верхнее поле
\geometry{bottom=2cm}% нижнее поле
\usepackage{cmap} % для кодировки шрифтов в pdf
\usepackage[T2A]{fontenc}
\pagestyle{empty} % нумерация выкл.
\begin{document}
	\begin{center}
		\textbf{АННОТАЦИЯ}
		
		\textbf{к дипломной работе на тему}
		
		\textbf{<<Моделирование распространения инфекционных заболеваний>> }
		
		\textbf{студента КФУ им. В.И. Вернадского Консманова А.В.}
	\end{center}

Работа включает: 41 страницу, 3 таблицы, 6 рисунков. Состоит из 2 разделов. Использованы 47 источников. 

\textbf{Цель работы:} анализ существующих моделей распространения инфекционных заболеваний и  создание нескольких различных собственных моделей, позволяющих прогнозировать распространение конкретных заболеваний на основе существующих статистических данных, реализовать полноценное приложение для использования  специалистами-эпидемиологами.

\textbf{Актуальность:} успехи современной медицины не позволяют мгновенно устранить вспышку инфекционного заболевания и точно предсказать место и характер этого события. Использование математических моделей позволяет предсказать характер вспышки и проанализировать ресурсы, необходимые для её локализации и устранения. Хотя на данный момент существует значительное  количество различных подходов к моделированию распространению инфекционных заболеваний, эти подходы требуют уточнения и реализации для конкретных заболеваний с последующим уточнением параметров.

\textbf{Объект исследования:} изучение распространения инфекционных заболеваний и соответствующих прогностических моделей.

\textbf{Предмет исследования:} построение и анализ моделей распространения инфекционных заболеваний.

\textbf{Методологическая основа} дипломной работы включает следующие методы: анализ, сравнение, моделирование процесса распространения инфекционного заболевания и метод разработки программного продукта.

По результатам исследования разработаны две модели с разными параметрами и подходами, которые могут быть использованы для прогнозирования характера потенциальной вспышки.
\end{document}
