\documentclass[a4paper,14pt,russian]{extreport}	%A4 бумага, 14 кегль, русский язык 
\usepackage{extsizes}
\usepackage[onehalfspacing]{setspace} % поулторный интервал %https://proft.me/2013/06/9/latex-ukazanie-mezhstrochnogo-intervala

\usepackage{cmap} % для кодировки шрифтов в pdf
\usepackage[T2A]{fontenc}
%\usepackage{pscyr}
%\usepackage{graphicx} % для вставки картинок
\usepackage{mathptmx} %поддержка textbf
\usepackage{makecell}
\usepackage{textcomp}
\usepackage{multirow} % улучшенное форматирование таблиц
\usepackage{ulem} % подчеркивания


%полужирный шрифт http://tostudents.ru/2009/12/08/smena-shriftov-v-latex-tekst-i-formuly/
\renewcommand{\rmdefault}{ftm} % Times New Roman
\usepackage[utf8]{inputenc}%включаем свою кодировку: koi8-r или utf8 в UNIX, cp1251 в Windows
%\usepackage[]{babel}	%больше поддержки русского языка 
\usepackage[english,russian, russianb]{babel}%используем русский и английский языки с переносами
\usepackage{amssymb,amsfonts,amsmath,mathtext,cite,enumerate,float} %подключаем нужные пакеты расширений

\usepackage[pdftex]{graphicx} %хотим вставлять в диплом рисунки?
\usepackage{cmap} % Улучшенный поиск русских слов в полученном pdf-файле
%\graphicspath{{images/}}%путь к рисункам
\usepackage{fancyhdr}%оформление нумерации 
\usepackage{tableof} %поддержка табличек
\usepackage{mathptmx}%
\usepackage{anyfontsize}% http://texblog.org/2012/08/29/changing-the-font-size-in-latex/
\usepackage{t1enc}%
\usepackage{cite}
\usepackage{graphicx}
\usepackage{xcolor} % цвет текста или фона https://tex.stackexchange.com/questions/136742/changing-background-color-of-text-in-latex
\graphicspath{{Images/}} %http://dkhramov.dp.ua/Comp.TexIncludeGraphics#.WwLdJa3sTMU
\DeclareGraphicsExtensions{.pdf,.png,.jpg} %http://dkhramov.dp.ua/Comp.TexIncludeGraphics#.WwLdJa3sTMU
%https://tex.stackexchange.com/questions/17734/cannot-determine-size-of-graphic
\makeatletter
\renewcommand{\@biblabel}[1]{#1.} % Заменяем библиографию с квадратных скобок на точку:
\makeatother

\usepackage{geometry} % Меняем поля страницы
%QUEST: 3cm или 2 ? Ес\usepackage{cmap} % Улучшенный поиск русских слов в полученном pdf-файлели 3, придется менять форматирование заголовка 
\geometry{left=3cm}% левое поле
\geometry{right=2cm}% правое поле
\geometry{top=2cm}% верхнее поле
\geometry{bottom=2cm}% нижнее поле





\renewcommand{\theenumi}{\arabic{enumi}}% Меняем везде перечисления на цифра.цифра
\renewcommand{\labelenumi}{\arabic{enumi}}% Меняем везде перечисления на цифра.цифра
\renewcommand{\theenumii}{.\arabic{enumii}}% Меняем везде перечисления на цифра.цифра
\renewcommand{\labelenumii}{\arabic{enumi}.\arabic{en\part{title}umii}.}% Меняем везде перечисления на цифра.цифра
\renewcommand{\theenumiii}{.\arabic{enumiii}}% Меняем везде перечисления на цифра.цифра
\renewcommand{\labelenumiii}{\arabic{enumi}.\arabic{enumii}.\arabic{enumiii}.}% Меняем везде перечисления на цифра.цифра
%http://fkn.ktu10.com/?q=node/6860
\addto\captionsrussian{\def\refname{Список  литературы}}
\renewcommand{\rmdefault}{ftm}
%NB: три команды ниже переопределяют некотрые шрифты  и дают поддержку жирного и прочиах текстов https://www.linux.org.ru/forum/general/4219163
\renewcommand{\rmdefault}{cmr} % Шрифт с засечками
\renewcommand{\sfdefault}{cmss} % Шрифт без засечек
\renewcommand{\ttdefault}{cmtt} % Моноширинный шрифт
\renewcommand*\thesection{\arabic{section}}
\renewcommand{\thefigure}{\thesubsection.\arabic{figure}} %Нумерация рисунков типа Рис 1.1. http://mydebianblog.blogspot.com/2008/12/latex_15.html
\renewcommand{\thetable}{\thesubsection.\arabic{figure}}
\usepackage{longtable}
\pagestyle{empty} % нумерация выкл.
\begin{document}
	\begin{center}
		\textbf{АННОТАЦИЯ}
		
		\textbf{к дипломной работе на тему}
		
		\textbf{<<Моделирование распространения инфекционных заболеваний>> }
		
		\textbf{студента КФУ им. В.И. Вернадского Консманова А.В.}
	\end{center}

Работа включает: 41 страницу, 3 таблицы, 6 рисунков. Состоит из 2 разделов. Использованы 47 источников. 

\textbf{Цель работы:} анализ существующих моделей распространения  инфекционных заболеваний и  создание нескольких различных собственных моделей, позволяющих прогнозировать распространение конкретных заболеваний на основе существующих статистических данных, реализовать полноценное приложение для использования  специалистами-эпидемиологами.

\textbf{Актуальность:} успехи современной медицины не позволяют мгновенно устранить вспышку инфекционного заболевания и точно предсказать место и характер этого события. Использование математических моделей позволяет предсказать характер вспышки и проанализировать ресурсы, необходимые для её локализации и устранения. Хотя на данный момент существует значительное  количество различных подходов к моделированию распространению инфекционных заболеваний, эти подходы требуют уточнения и реализации для конкретных заболеваний с последующим уточнением параметров.

\textbf{Объект исследования:} изучение распространения инфекционных заболеваний и соответствующих прогностических моделей.

\textbf{Предмет исследования:} построение и анализ моделей распространения инфекционных заболеваний.

\textbf{Методологическая основа} дипломной работы включает следующие методы: анализ, сравнение, моделирование процесса распространения инфекционного заболевания и метод разработки программного продукта.

По результатам исследования разработаны две модели с разными параметрами и подходами, которые могут быть использованы для  прогнозирования характера потенциальной вспышки.
\end{document}
